\documentclass{cam-thesis}
\usepackage{setspace}
\onehalfspacing
\usepackage{graphicx}
\usepackage[utf8]{inputenc}

\usepackage{amsmath}
\usepackage{amssymb}
\usepackage{floatpag}
\usepackage{bm}
\usepackage{algorithm2e}
\usepackage[nameinlink]{cleveref}
\usepackage[style=british]{csquotes}
\usepackage[T1]{fontenc}
\usepackage{hyperref}
\usepackage{textcomp} % provide symbols
\usepackage{todonotes}
\usepackage{xr}
\usepackage{pdflscape}
\usepackage{afterpage}
\usepackage{caption}
% \usepackage{microtype}
\usepackage[UKenglish]{babel}

% Diagrams
\usepackage{tikz}
\usetikzlibrary{arrows,positioning,automata,fit}

% Macros for common abbreviations to get the spacing right
% See: https://stackoverflow.com/questions/3282319/correct-way-to-define-macros-etc-ie-in-latex
\usepackage{xspace}
\makeatletter
\DeclareRobustCommand\onedot{\futurelet\@let@token\@onedot}
\def\@onedot{\ifx\@let@token.\else.\null\fi\xspace}
\def\eg{e.g\onedot} \def\Eg{\emph{E.g}\onedot}
\def\ie{i.e\onedot} \def\Ie{\emph{I.e}\onedot}
\def\cf{c.f\onedot} \def\Cf{\emph{C.f}\onedot}
\def\etc{etc\onedot} \def\vs{\emph{vs}\onedot}
\def\wrt{w.r.t\onedot} \def\dof{d.o.f\onedot}
\def\etal{et al\onedot}
\makeatother

% Generic maths commands
\def\reals{\mathbb{R}}
\def\nats{\mathbb{N}}
\def\sampSpace{\mathcal{X}}
\def\dist{\sim}
\DeclareMathOperator{\E}{\mathbb{E}}
\DeclareMathOperator{\V}{\mathbb{V}}
\DeclareMathOperator{\I}{\mathbb{I}}
\DeclareMathOperator{\prob}{\mathbb{P}}
\DeclareMathOperator{\p}{\pi}
\DeclareMathOperator{\var}{\mathbb{V}}
\DeclareMathOperator{\indicator}{\mathbb{I}}
\DeclareMathOperator{\cov}{Cov}
\DeclareMathOperator{\cor}{Cor}
\DeclareMathOperator{\logit}{logit}
\DeclareMathOperator{\Ber}{Bernoulli}
\DeclareMathOperator{\Bin}{Binomial}
\DeclareMathOperator{\Poi}{Poisson}
\DeclareMathOperator{\BetaDist}{Beta}
\DeclareMathOperator{\Exponential}{Exponential}
\DeclareMathOperator{\NBr}{NegBin}
\newcommand{\NBc}{\NBr_{c}}
\newcommand{\NBs}{\NBr_{s}}
\DeclareMathOperator{\BB}{BetaBin}
\DeclareMathOperator{\GamDist}{Gamma}
\DeclareMathOperator{\MN}{Multinomial}
\DeclareMathOperator{\N}{N}
\DeclareMathOperator{\LN}{LN}
\DeclareMathOperator{\LKJ}{LKJ}
\DeclareMathOperator{\expit}{expit}
\newcommand\matr{\bm}
\newcommand\set{\mathcal}
\renewcommand{\vec}{\bm}
\newcommand{\ssep}{:}

% Thesis-specific maths commands
\newcommand{\dmax}{d_\text{max}}
\newcommand{\psens}{p_\text{sens}}
\newcommand{\ntot}{n_\text{tot}}
\newcommand{\ndet}{n_\text{d}}
\newcommand{\nnodet}{n_\text{u}}
\newcommand{\pnodet}{p_\text{u}}
\newcommand{\Npop}{N_\text{pop}}
\newcommand{\Ncis}{N_\text{CIS}}
\newcommand{\ncis}{\vec{n_\text{CIS}}}
\newcommand{\na}{\vec{n_\text{a}}}
\newcommand{\pcis}{\vec{p_\text{CIS}}}
\newcommand{\sched}{\mathbb{T}}
\newcommand{\rate}{\mathfrak{R}}
\newcommand{\R}{\mathcal{R}_{0}}
\newcommand{\vlod}{v_\text{lod}}

\newcommand\citePersonalComms[1]{(#1, personal communication)}


\title{Estimating SARS-CoV-2 transmission in England from randomly sampled prevalence surveys}
\author{Joshua Blake}
\date{\today}
\subjectline{MRC Biostatistics Unit}
\college{Jesus College}
\collegeshield{CollegeShields/jesus}
% \keywords{one two three}

%% Submission date [optional]:
% \submissiondate{November, 2042}

%% You can redefine the submission notice [optional]:
% \submissionnotice{A badass thesis submitted on time for the Degree of PhD}

% Only number labelled equations
% \forlistloop workaround due to incompatability with biblatex
% See: https://tex.stackexchange.com/questions/220268/biblatex-and-autonum-dont-work-together
\usepackage{etoolbox} % get the good meaning of \forlistloop
\let\etoolboxforlistloop\forlistloop % save the good meaning of \forlistloop
\usepackage{autonum}
\let\forlistloop\etoolboxforlistloop % restore the good meaning of \forlistloop
\makeatletter
\let\blx@noerroretextools\empty
\makeatother
\usepackage[backend=biber,style=authoryear,uniquename=false,sortcites=true]{biblatex}
\addbibresource{references.bib}

\usepackage{subfiles} % Best loaded last in the preamble
\addbibresource{references.bib}
\externaldocument[E-]{\subfix{thesis}}

\abstract{%
    TODO
}

\acknowledgements{%
  TODO
}

\usepackage{geometry}
\setlength{\marginparwidth}{3cm}
\geometry{margin=3.8cm}

\begin{document}

\frontmatter{}

\listoffigures

\chapter{Introduction} \label{intro}

\section{The COVID-19 pandemic and ongoing burden}

\section{SARS-CoV-2 and COVID-19}

Viral loads are infrequently measured exactly.
The most common form of data obtained is cycle threshold (Ct) values from real-time quantitative  reverse-transcriptase polymerase chain reaction (PCR) tests.
The Ct value is proportional to the negative log of the viral load which is convenient when modelling log viral load as a piecewise linear function, since Ct values themselves are piecewise linear under this assumption.
When a host's viral load is too low (below the limit of detection of the PCR test), no Ct value is available and hence viral load can be considered censored at the limit of detection.
The relationship between Ct and viral load is noisy and can vary based on many factors including quality of swab obtained, primers used (even between batches), and the PCR system set-up~\autocites{dahdouhCt,hanRTPCR}.

\section{Thesis outline}

\subfile{biology-data}

\subfile{incidence-prevalence}

% We wish to estimate the distribution of time for which an individual is positive for.
% An individual could have multiple infections, and hence I refer to \emph{infection episodes}.

% I define an individual as \emph{detectable} if they would test positive if tested, in the absence of misclassification error (which is dealt with in \cref{E-imperf-test}).
% Episode $i$'s \emph{duration of positivity}, $D_i$ is the number of days that they are detectable for; I use time discretised to days, therefore, the duration is at least 1 day.
% Note that the relationship between positivity (\ie the result of a PCR test) and infectiousness (\ie an individual's ability to pass on the virus) is complex~\autocites{lascolaViral}{singanayagamDuration}, and beyond the scope of this thesis.
% I define an episode as starting at time $B_i$ and ends at $E_i$, hence $D_i = E_i - B_i + 1$.

% An episode is \emph{detected} when they test positive at least once, if this occurs; episode $i$ is detected at time $t$ if and only if: $i$ is detectable at $t$, $i$ is tested at $t$, and the test returns positive (a false negative does not occur).

% The survival time function, parameterised by $\theta$, is the probability that an episode lasts at least $t$ days: $S_\theta(t) = \prob(D_i \geq t \mid B_i = b, \theta)$, where $D_i$ and $B_i$ are the duration and starting time of episode $i$ respectively.
% I make the standard assumption that $D_i$ and $B_i$ are independent and hence: $S_\theta(t) = \prob(D_i \geq t \mid \theta)$.I utilise survival analysis for analysing this data.
% Survival analysis is the area of statistics concerned with estimating the distribution of the times between events.
% It is broadly applicable to many domains, such as the time-to-failure for a mechanical system or the effect of a treatment on time spent in hospital.
% In the context of this thesis, the distribution to estimate is that of duration of positivity.
% The \emph{initiating event} is when the duration starts, \ie becoming detectable.
% The \emph{terminating event} is when the duration ends, \ie becoming no longer detectable.

\subfile{ATACCC}

\subfile{cis-perfect-testing}

\subfile{cis-imperfect-testing}

\chapter{Estimating transmission using backcalculation} \label{backcalc}

\subfile{SEIR}

\chapter{Conclusion}

\listoftodos

\printbibliography

\appendix

\subfile{distributions}

\subfile{ATACCC-appendix-original-analysis}

\chapter{CIS episode definitions} \label{episode-def}

\todo[inline]{Get this from SARAH}

\end{document}