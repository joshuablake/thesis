\documentclass[thesis.tex]{subfiles}

\begin{document}
\ifSubfilesClassLoaded{
    \setcounter{chapter}{7}
}

\chapter{Conclusions} \label{conclusion}

Pandemics will continue to pose a significant threat to public health for the foreseeable future.
Therefore, there is a public health need to respond to pandemics.
Such a response needs to be informed by monitoring.
In this thesis, I have investigated one method of monitoring pandemics: prevalence surveys.

\section{Main findings}

In \cref{E-intro:sec:aims}, I posed the central question for this thesis: can prevalence surveys alone be used to infer incidence and transmission, and if so how?

In \cref{E-transmission}, I showed that this it is possible by appying two models.
The differing models, one statistical in nature and one mechanistic, each had their own strengths and weaknesses.
The statistical approach, based on the backcalculation framework, estimates incidence and makes fewer assumptions about the infection process.
However, the estimates are very uncertain.
The mechanistic approach, within the compartmental modelling framework, reduces the uncertainty and allows estimation of the components of transmission by making stronger assumptions about the infection process.

These results highlighted the heterogeneity across regions and age groups in England.

These models relied on the theoretical work of \cref{E-inc-prev}.
This work developed the backcalculation framework for the context of prevalence surveys for SARS-CoV-2.
The major insight was the justification of the deterministic model framework and the importance of the distribution of the duration of positivity.

In \cref{E-perf-test,E-imperf-test}, I developed methodology to estimate the distribution of the duration of positivity from the CIS prevalence survey.
Therefore, showing that this important parameter can be estimated from prevalence surveys.
While I did make use of external estimates, from \cref{E-ATACCC}, this was to address weaknesses of the CIS study design that could be addressed in future studies (see \cref{conclusion:sec:future-work}).

Both approaches assumed that the distribution of the duration of positivity was perfectly known.

Estimating this distribution was the focus of the majority of this thesis.


\section{Future work} \label{conclusion:sec:future-work}

The feasibility of jointly modelling duration and incidence was not investigated in this thesis.
The model for duration in \cref{E-perf-test,E-imperf-test} estimated the total number of infections in the cohort but assumed these occurred uniformly over the duration of the survey.
Relaxing the uniformity assumption, would give incidence estimation.
The issue of including covariates to control for non-response bias, encountered in \cref{E-transmission}, as well as the parameterisation of the shape of the incidence curve would be major challenges in this extension.
Modelling frameworks could be adopted from the existing literature~\autocite[e.g.][]{taffeJoint,haySerosolver} used approximations (\eg empirical Bayes), smaller sample sizes, or high-performance computing to address these challenges.
However, the CIS is larger and incidence estimation desired at a more granular level than previously encountered.
Furthermore, high-performance computing cannot be used within the SRS.

A common theme in this thesis was computational challenges.
The challenges could be addressed by more efficient MCMC algorithms, approximate methods, or alternate computing methods (\eg the use of GPUs).
Alternate computing methods would require appropriate hardware within the SRS, which should be considered when choosing the design of future TREs.

The CIS was a household-based survey, but the household structure was not considered in this thesis.
The household-based design is likely to induce clustering in results.
Therefore, the uncertainty in my results might be underestimated.
While the beta-binomial likelihood used in \cref{E-transmission} should capture this, it does not consider the clustering between different strata.
Other chapters did not consider clustering at all.
Similarly, the treatment of a longitudinal survey as a cross-sectional survey in \cref{E-perf-test} may have underestimated the uncertainty in the results.
More sophisticated incorporation of clustering, for instance using random effects for households and/or individuals, could be considered in future work.

The challenges in estimating the distribution of the duration of positivity originated in the design of the studies used.
Exploring the design of future studies to better estimate the distribution of the duration of positivity would be beneficial.
A particularly promising direction would be intensive follow-up of individuals following a positive test (\eg daily testing) until they have two or more negative tests.
Intensive follow-up would increase the precision by which the end of the infection episode is known.
Furthermore, it reduces the impact of false negatives because they are likely to be followed by a true positive.
I have developed software that could be used to perform simulation studies of alternative designs.

Other design questions are also posed by the work in this thesis.
For example, whether a representative sample of the population is optimal, or if certain groups should be oversampled.
Certain groups, such as younger individuals, were more likely to be infected.
It is plausible that sampling in proportion to incidence would be beneficial.
This is especially true if policy questions (\eg closing of schools) are of importance.
These questions could be investigated through simulation studies.

\ifSubfilesClassLoaded{
  \listoftodos
}{}

\end{document}