\documentclass[thesis.tex]{subfiles}

\begin{document}
\chapter{Statistical background} \label{incidence-prevalence}
%\todo[inline]{This chapter is pretty short. Not sure if this matters or it needs to merged with another chapter? For example, one background chapter split into epi/biology and stats.}

This chapter explains incidence and prevalence, and the relationship between these quantities.
These quantities have been of interest as long as disease has been studied, and estimating the prevalence from incidence is the fundamental motivation for this thesis.
Their exact definition can vary throughout the literature, here I follow the definitions of the seminal paper \textcite{freemanPrevalence}.
As I will explain, the duration of infection is the quantity that relates them.

Timescale underlies all these quantities.
Throughout this thesis, time will generally\todo{revisit phrasing once finished - do I actually use continuous time anywhere (probably in SEIR model)} be considered discrete in units of days.
Most epidemiological data is only available at daily granularity or, when more granular data exists, is often unreliable.
Furthermore, within-day patterns produce variations in transmission (\eg due to sleep) and data (\eg due to logistical considerations around collection); these variations are not of interest in most contexts.

\section{Definitions}

\emph{Incidence} at time $t$, $Z_t$, is the number of people in a given population who become infected at time $t$ (I use $Z_t$ rather than $I_t$ to avoid confusion with the convention that $i$ is an indexing variable).
Incidence expressed as a proportion of the population, $Z_t/N$ where $N$ is the number of people in the population, is the \emph{incidence rate}.
As discussed in \cref{E-intro}\todo{might want to point to a specific section}, infection events are not directly observable and must be inferred from other data.
Despite this issue, it remains the single most important quantity for informing the response to an epidemic or pandemic.

\emph{Prevalence} of an infectious disease at time $t$, $P_t$, is the number of people in a given population currently infected at time $t$, assumed constant over the short period $t$.
Each of these individuals is known as a \emph{prevalent} individual.
The \emph{prevalence proportion} is the same quantity expressed as a proportion of the population.
For SARS-CoV-2, there are several definitions of infected.
An infected individual could mean: one that is infectious; one that is symptomatic; or one that, if they were tested, would return a positive PCR test if tested (discussed in\todo{ref appropriate section}).
In this thesis, prevalence is defined as the number of people in the population who would return a positive PCR test (if everyone was tested) at time $t$.

\section{Backcalculation}

\emph{Backcalculation} is the procedure for estimating incidence based on a measurement of prevalence.
\Textcite{brookmeyerMethod} developed it in the context of HIV (human immunodeficiency virus) and AIDS (acquired immune deficiency syndrome).
HIV is transmitted sexually or through blood and causes the disease AIDS.
It has killed 40 million people, peaking at 2 million annual deaths~\autocite{unaids2023}.
In this context, backcalculation estimates the incidence of HIV infection from: data on the number of AIDS diagnoses, and the distribution of the time between infection and AIDS diagnosis~\autocites{brookmeyerBackcalculation}{brookmeyerMeasuring}.
AIDS is incurable, therefore, AIDS prevalence is equivalent to cumulative incidence.

Three processes are involved in backcalculation: the \emph{infection process}, the \emph{prevalence process} and the \emph{observation process}.
The infection process determines incidence.
The prevalence process describes the relationship between incidence and prevalence.
The observation process describes the relationship between prevalence and the observed data.

The original formulations of backcalculation neglected the observation process.
AIDS is almost always diagnosed and recorded in developed countries.
That is, the observations of AIDS diagnoses are the same as the true incidence of AIDS.
An exception is that reporting delay may be important~\autocite{paganoHIV}.
Reporting delay is not relevant to the studies in this thesis, and I ignore it in this and further discussion.
Therefore, two processes need to be considered: infection and prevalence.

For SARS-CoV-2, measurements of prevalence are noisy because they are based on population sampling.
The sampling is described as the observation process.
The observation process noise dominates the noise of the prevalence process (see \cref{incidence-prevalence:sec:observation-process}).
Therefore, the latter is negligible and again two processes need to be considered: infection and observation.

\section{Infection process}

The infection process is a point process, where each infection is an event.
A simple and commonly-used model is the time-inhomogenous Poisson process~\autocites{brookmeyerMethod}{paganoHIV}{rosenbergBackcalculation}{brookmeyerBackcalculation}.
That is, a Poisson process where the intensity varies over time.
The target of inference can be either the process's intensity or the counts of the specific realisation that occurred.
The realised counts are normally of greater interest because we care about the realised epidemic, not what may happen if there were future realisations of the same epidemic.

Two features of a Poisson process can be violated for epidemics.
Epidemics commonly exhibit overdispersion and have feedback loops.
Overdispersion means that the variance in the number of infections in a time interval is higher than its mean.
In a Poisson process, counts in a time interval have a Poisson distribution.
Therefore, the mean and variance are equal.
Feedback loops mean that the stochastically low or high numbers of infections, possibly due to overdispersion, affect the future number of infections.
In a Poisson process, the intensity (mean number of events per unit time) is not influenced by the realisation of the process.
Both of these features violate the assumptions of a Poisson process.
Despite these issues, satisfactory inference on the incidence can still be produced, although the uncertainty in the estimates may be understated~\autocite{beckerDependent}.

A common cause of overdispersion is super-spreading events~\autocite{lloyd-smithSuperspreading}.
A super-spreading event is an event where many more people than expected are infected.
Coronaviruses, such as SARS-CoV-2, have particularly high overdispersion~\autocites{endoEstimating}{adamClustering}{mccloskeySARS}.
Overdispersion could be incorporated by modelling the counts in a given time interval as a negative binomial, rather than Poisson, distribution.
The mean of these distributions would be equal, but the negative binomial introduces an additional overdispersion parameter, commonly notated $k$.
These distributions are related in that the negative binomial tends to Poisson as $k\to\infty$.
Generally, $k<1$ indicates significant levels of overdispersion.
Estimates for SARS-CoV-2 indicate $k \approx 0.1$~\autocite{endoEstimating}, and can be increased by interventions such as lockdowns~\autocites{quiltyReconstructing}{quiltyUnderstanding}.
However, the negative binomial distribution loses much of the mathematical convenience of a Poisson distribution.
For a negative binomial process, the count's standard deviation scales on the order of the mean.
Conversely, for a Poisson process, the standard deviation is on the order of the square root of the mean.
A particular implication of this is that the coefficient of variation (the standard deviation divideed by the mean) tends to zero for a Poisson process, but tends to a constant for a negative binomial process.

The independence assumption is violated because if the number of infections is stochastically high, this feeds back to the process intensity at future times.
This can be seen simply in the event of a super-spreading event.
If a super-spreading event occurs, then many more individuals than suggested by the process intensity will be infected.
These additional infections will propagate the epidemic further, increasing the intensity at future times.

An alternative to a Poisson process is \emph{mechanistic models}.
Mechanistic models explicitly describe the population and transmission between individuals.
They vary greatly in their details and realism~\autocite{murilloMultiscale}.
Mechanistic models are discussed further in \cref{E-SEIR}.

% In \emph{deterministic backcalculation}, the observed prevalence, $x_t/n_t$, is assumed to be equal to $P_t/N$.
% However, this is often a poor approximation.
% \emph{Statistical backcalculation} retains the sampling distribution of $x_t$ (or an approximation of it, such as a Poisson).

% Define the first day of an infection episode in individual $i$ as $B_i$, and assume that the probability of $i$ being infected multiple times within the period of interest is negligible.
% The time between being infected and first being detectable is short (see\todo{ref relevant part}), and therefore I often assume that $B_i$ is the same as the time of infection.
% I keep this assumption for the remainder of this chapter.


% Backcalculation makes use of this relationship to estimate incidence from prevalence, assuming that the duration is known.

% Deterministic backcalculation assumes that the variance in the population is negligible, and therefore that the prevalence is equal to the mean prevalence.
% This is justified because the variance due to sampling is much larger than the variance due to the true prevalence.

\section{Prevalence process}

The relationship between incidence and prevalence can be derived by considering the probability that an individual is prevalent at any time following an infection.
The period the individual is prevalent for is the \emph{infection episode}, which I will index with $i$, associated with that infection event.
Denote the time that the infection episode begins as $B_i$ and the time it ends as $E_i \geq B_i$.
The individual is prevalent from $B_i$ to $E_i$ inclusive.
The duration of the infection episode, the number of days for which they are prevalent, is $D_i = E_i - B_i + 1$.
In this thesis, it is assumed that the $D_i$ for each infection episode is independent and $\prob(D_i = d) = f_d$ for some pmf $f$ over $1, 2, \dots, d_\text{max}$.

From this, we can derive the relationship between incidence and prevalence.
Denote by $R_{t,t'}$ be the number of infection episodes that begin at time $t$ and end at time $t'$.
For each infection beginning at $t$, the probability that it ends at $t'$ is $f_d(t' - t + 1)$.
Since these durations are assumed independent, the number of episodes of each duration is multinomially distributed~\autocite{paganoHIV}.
Therefore:
\begin{align}
\begin{bmatrix}
  R_{t,t} \\ R_{t,t+1} \\ \vdots \\ R_{t,t+d_\text{max}-1}
\end{bmatrix} \mid Z_t
\sim \MN \left(
  Z_t, 
  \begin{bmatrix}
    f_d(1) \\ f_d(2) \\ \vdots \\ f_d(d_\text{max})
  \end{bmatrix}
\right).
\end{align}
Now, denote by $P_{t_z,t_p}$ the number of episodes beginning at time $t_z$ that are prevalent at time $t_p$ (where $t' \geq t$).
This is $Z_t$ minus those that have recovered:
\begin{align}
    P_{t_z,t_p} = \begin{cases}
      0 &t_p < t_z\\
      Z_t - \sum_{i=0}^{t_p-t_z-1} R_{t_z,t_z+i} &t_z \leq t_p < t_z + d_\text{max}\\
      0 &t_p \geq t_z + d_\text{max}.
  \end{cases} \label{incprev:eq:Ptt-to-R}
\end{align}
Finally, the prevalence at $t$ can be expressed as the total number of people prevalent at time $t$, regardeless of when they were infected:
\begin{align}
  P_t
  &= \sum_{t_z=\infty}^\infty P_{t_z,t} \\
  &= \sum_{i=0}^{d_\text{max}-1} P_{t-i,t} \label{incprev:eq:Pt-to-Ptt}.
\end{align}

In general, all these random variables will be dependent for an infectious disease.
This is because the probability of being infected at time $t$ depends on the number of infectious individuals at time $t$, which itself depends on the number of individuals infected before time $t$.
However, if we condition on the vector $\vec{Z} = (Z_1, Z_2, \dots)^T$, then we will induce some independence.
In particular, because the duration of individuals is independent, $R_{t_1,t_2}$ and $R_{t_1',t_2'}$ will be independent if $t_1 \neq t_1'$; importantly, this holds even if $t_2 = t_2'$.

Only $\E(P_t \mid \vec{Z})$ will be relevant (see \cref{incidence-prevalence:sec:observation-process}), which can be derived from its constituent parts.
Start by considering the recoveries.
The sum of recoveries of infections that occurred on the same day $t$ is the sum of multinomial cell probabilities, and hence binomially distributed~\autocite{alamAnalysis}.
Specifically:
\begin{align}
  \sum_{i=0}^{t'} R_{t,t+i} \mid \vec{Z} &\sim \text{Binomial}(Z_t, F_d(t'+1)).
\end{align}
where $F_d(t) = \sum_{i=1}^t f_d(t) = \prob(D \leq t)$ is the CDF of $D$.
Therefore:
\begin{align}
  \E(\sum_{i=0}^{t'} R_{t,t+i} \mid \vec{Z}) = Z_t F_d(t'+1) \label{incprev:eq:EsumRt}
\end{align}
Now consider $P_t$:
\begin{align}
\E(P_t \mid \vec{Z})
  &= \E\left(\sum_{i=1}^{d_\text{max}-1} P_{t-i,t} \mid \vec{Z} \right) &\text{by \cref{incprev:eq:Pt-to-Ptt}} \\
  &= \E\left(\sum_{i=1}^{d_\text{max}-1} \left( Z_{t-i} - \sum_{j=0}^{i-1} R_{t-i,t-i+j} \right) \mid \vec{Z} \right) &\text{by \cref{incprev:eq:Ptt-to-R}}\\
  &= \sum_{i=1}^{d_\text{max}-1} \left( Z_{t-i} - \sum_{j=0}^{i-1} \E( R_{t-i,t-i+j} \mid \vec{Z}) \right) \\
  &= \sum_{i=1}^{d_\text{max}-1} \left( Z_{t-i} - Z_{t-i} F(i) \right) &\text{by \cref{incprev:eq:EsumRt}}\\
  &= \sum_{i=1}^{d_\text{max}-1} Z_{t-i} (1 - F(i)) \\
  &= \sum_{i=1}^{d_\text{max}-1} Z_{t-i} S(i+1)
\end{align}
where $S(i+1) = 1-F(i) = \prob(D \geq i+1)$ is the \emph{survival function} of $D$ (see \cref{incidence-prevalence:sec:duration}).

% The probability of being positive can be modelled in different ways depending on what assumptions are most reasonable.
% The most important assumption from a statistical perspective is the whether the probability of testing positive on a given day is independent conditional on the infection time.

% \todo[inline]{Is the discussion of this model just a distraction?}
% Assuming that the probability of testing positive depends only on the infection time implies no individual variation.
% This model is formalised as $Pr(Z_i(t) = 1 \mid B_i, Z_i(1), Z_i(2) \dots) = Pr(Z_i(t) \mid B_i) = p_{t-B_i}$, that is the probability of individual $i$ testing positive $t$ days after infection depends only on the infection start time and no other quantities.
% A slight relaxation of this model is introducing covariates, allowing some variation based on an individual's characteristics.

% The model with the most dependence between times, while remaining realistic, is a multi-state model.
% Therefore, if an individual has tested negative before the present time $t$ but after $B_i$ then they will test negative at time $t$.
% The complexity of the transitions between positive and negative can be very complex.
% I consider a slight extension to allow for the possibility of false negatives, meaning a negative test between $B_i$ and $E_i$, and that this probability may change.
% This leads to the following model.
% \begin{align}
%   \prob(Z_i(t) = 1 \mid B_i, \leq E_i) &= \begin{cases}
%     \psens(t - B_i) &B_i \leq t \leq E_i\\
%     0 &\text{otherwise}
%   \end{cases}\\
%   \prob(E_i = t \mid B_i) = f_i(B_i)
% \end{align}
% Conditional on both $B_i$ and $E_i$, then $Z_i(t)$ and $Z_i(t')$ are independent for $t \neq t'$.
% However, unlike the previous model, conditioning only on $B_i$ is not sufficient for $Z_i(t)$ and $Z_i(t')$ to be independent because each $Z_i(t)$ provides information on $E_i(t)$ (in particular, $Z_i(t) = 1$ implies $E_i > t$).
% For SARS-CoV-2, the latter model is more appropriate, although, as we shall see, the assumption that $\psens$ is independent of $t$ is violated.

The equation derived here is a discrete convolution equation, specifically a Volterra equation of the first kind~\autocite{brookmeyerBackcalculation}.
Clearly, knowledge of the survival function, $S$, will be important; I discuss this further in \cref{incidence-prevalence:sec:duration}.

\section{Observation process} \label{incidence-prevalence:sec:observation-process}

% A \emph{prevalence survey} is a survey that tests a sample of the population for the presence of an infection.
% Series of prevalence surveys are the form of measuring prevalence that this thesis is concerned with.
% This thesis aims to infer the incidence rate from prevalence surveys.
% I use Bayesian inference, therefore the data contributes through the likelihood.

Prevalence surveys, as described in\todo{insert ref}, sample individuals at random from the population of interest and test if they are prevalence.
Therefore, I model the number of positive tests on day $t$ as a random variable $X_t$ based on a sample of $n_t$ individuals.
The $X_t$s are conditionally independent, given the population prevalences at each time, $\vec{P}_t$. 
%For now, I assume that the sample is chosen uniformly at random from the population of interest.
%Assume that the surveys are independent of each other and $\vec{Z}$, conditional on the population prevalence at that time.
%The population prevalence is a latent quantity.
From these assumptions, $X_t \mid \vec{P} \sim \Bin(n_t, P_t / N)$.
I base inference on the likelihood, $p(\vec{x} \mid \vec{Z}) = \int p(\vec{x} \mid \vec{P}) p(\vec{P} \mid \vec{Z}) d\vec{P} = \int \prod_t p(x_t \mid P_t) p(\vec{P} \mid \vec{Z}) d\vec{P}$, where $\vec{x} = (x_1, x_2, \dots)^T$.

For most prevalence surveys, the uncertainty arising from the distribution $\vec{P} \mid \vec{Z}$ is negligible.
This can be seen by considering the variance in the likelihood.
Assume $x_t \sim \Bin(n_t, P_t / N)$, where $N$ is the size of the population.
Therefore, by the law of total variance:
\begin{align}
  \var\left( x_t \mid \vec{n}, \vec{Z} \right)
    &= \var\left[\E\left( x_t \mid \vec{n}, \vec{Z}, \vec{P} \right) \mid \vec{n}, \vec{Z} \right] + \E\left[\var\left( x_t \mid \vec{n}, \vec{I}, \vec{P} \right) \mid \vec{n}, \vec{Z} \right] \\
    &= \var\left( n_t \frac{P_t}{N} \mid \vec{n}, \vec{Z} \right) + \E\left( n_t \frac{P_t}{N} \left(1 - \frac{P_t}{N} \right) \mid \vec{n}, \vec{Z} \right) \\
    &= \left( \frac{n}{N} \right)^2 \var\left(P_t \mid \vec{n}, \vec{Z} \right) + \frac{n_t}{N} \E\left(P_t \mid \vec{n}, \vec{Z} \right)  - \frac{n_t}{N^2} \E\left(P_t^2 \mid \vec{n}, \vec{Z} \right).
\end{align}
For large populations and reasonable sample sizes, $n_t << N$ and $\E(P_t) \approx \var(P_t)$, the first term is negligible.
Hence, assuming that $\var(\vec{P}) = \matr{0}$ is a reasonable approximation.
This is equivalent to assuming that $\vec{P}$ is a deterministic function of $\vec{Z}$, $\vec{P} = \E( \vec{P} \mid \vec{Z})$.


\section{Estimating duration} \label{incidence-prevalence:sec:duration}

\todo[inline]{This section fills a bit out of place, could maybe be a more general conclusion? Although these ideas and notation should probably stay...}

Robust, unbiased prevalence estimates already exist (see\todo{ref prevalence estimates}), and incidence estimates are the most useful (see\todo{ref intro with importance of incidence}).
Therefore, a large portion of this thesis concerns itself with estimating the duration of PCR positivity.

The distribution of durations can be expressed in several equivalent ways.
Let $D_i$ be the random variable for the duration of episode $i$, then the distribution of interest is that of $D_i$.
I will assume that each $D_i$ is an independent and identically distributed realisation of a random variable $D$; extending the methodology to account for covariates is left to further work.
For an infection in discrete time, the duration is a non-negative integer, \ie $D \in \{1, 2, \dots\}$.
The first way of expressing the distribution is as the standard probability mass function (pmf), $f_{D}(d) = \prob(D = d)$; or cumulative density function (CDF), $F_{D}(d) = \prob(D \leq d) = \sum_{i=0}^d f_{D}(i)$.
The \emph{survival function} gives the probability that the duration is at least $d$ days: $S(d) = \prob(D \geq d) = 1 - F_{D}(D - 1)$ with $S(1) = 1$.
The \emph{hazard} at time $t$ is the probability of recovering at time $t$ conditional on having survived to time $t$: $\lambda(t) = \prob(D = t \mid D \geq t) = f_{D}(d) / S(d)$.
Throughout this thesis, the survival function will bne the primary quantity of interest, and it will normally be parameterised in terms of the hazard; these two quantities are related by $S(t) = \prod_{i=1}^{t-1} (1 - \lambda(i))$.

Two approaches are commonly used to estimate durations: either modelling an underlying biomarker or modelling the duration directly~\autocite{sweetingEstimating}.
For SARS-CoV-2, both approaches would be based on the results of PCR testing.
Modelling a biomarker would mean modelling the viral load, measured as a Ct value (see \cref{E-ATACCC}).
Modelling the duration would consider only the binary result at each test (positive or negative); in this thesis a \emph{survival analysis} framework is adopted (see \cref{E-perf-test,E-imperf-test}).
% Survival analysis is the area of statistics concerned with estimating the distribution of the times between two events.
Other related approaches, such as a multi-state model~\autocite{jacksonMSM}, could also be considered (multi-state models can be viewed as a generalisation of survival analysis).
Therefore, the biomarker approach considers more information per observation, however, a model for the biomarker is required which generally requires stronger assumptions.
Modelling a biomarker uses more information per observation, but requires stronger assumptions.

\end{document}