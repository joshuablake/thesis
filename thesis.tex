\documentclass{cam-thesis}
\usepackage{setspace}
\onehalfspacing
\usepackage{graphicx}
\usepackage[utf8]{inputenc}

\usepackage{amsmath}
\usepackage{amssymb}
\usepackage{bm}
\usepackage[nameinlink]{cleveref}
\usepackage[style=british]{csquotes}
\usepackage[T1]{fontenc}
\usepackage{hyperref}
\usepackage{textcomp} % provide symbols
\usepackage{todonotes}
\usepackage{xr}
% \usepackage{microtype}

% Macros for common abbreviations to get the spacing right
% See: https://stackoverflow.com/questions/3282319/correct-way-to-define-macros-etc-ie-in-latex
\usepackage{xspace}
\makeatletter
\DeclareRobustCommand\onedot{\futurelet\@let@token\@onedot}
\def\@onedot{\ifx\@let@token.\else.\null\fi\xspace}
\def\eg{e.g\onedot} \def\Eg{\emph{E.g}\onedot}
\def\ie{i.e\onedot} \def\Ie{\emph{I.e}\onedot}
\def\cf{c.f\onedot} \def\Cf{\emph{C.f}\onedot}
\def\etc{etc\onedot} \def\vs{\emph{vs}\onedot}
\def\wrt{w.r.t\onedot} \def\dof{d.o.f\onedot}
\def\etal{et al\onedot}
\makeatother

% Generic maths commands
\def\reals{\mathbb{R}}
\def\nats{\mathbb{N}}
\def\sampSpace{\mathcal{X}}
\def\dist{\sim}
\DeclareMathOperator{\E}{\mathbb{E}}
\DeclareMathOperator{\V}{\mathbb{V}}
\DeclareMathOperator{\prob}{\mathbb{P}}
\DeclareMathOperator{\var}{\mathbb{V}}
\DeclareMathOperator{\indicator}{\mathbb{I}}
\DeclareMathOperator{\cov}{Cov}
\DeclareMathOperator{\logit}{logit}
\DeclareMathOperator{\Ber}{Bernoulli}
\DeclareMathOperator{\Bin}{Binomial}
\DeclareMathOperator{\Poi}{Poisson}
\DeclareMathOperator{\Beta}{Beta}
\DeclareMathOperator{\NegBin}{NegBin}
\DeclareMathOperator{\GamDist}{Gamma}
\def\matr{\bm}
\def\set{\mathcal}
\renewcommand{\vec}{\bm}

% Thesis-specific maths commands
\newcommand{\psens}{p_\text{sens}}

\newcommand{\citePersonalComms}{(personal communication)}


\title{Estimating the incidence of SARS-CoV-2 from prevalence data}
\author{Joshua Blake}
\date{\today}
\subjectline{MRC Biostatistics Unit}
\college{Jesus College}
\collegeshield{CollegeShields/jesus}
% \keywords{one two three}

%% Submission date [optional]:
% \submissiondate{November, 2042}

%% You can redefine the submission notice [optional]:
% \submissionnotice{A badass thesis submitted on time for the Degree of PhD}


\usepackage[backend=biber,style=authoryear,uniquename=false]{biblatex}

\usepackage{subfiles} % Best loaded last in the preamble
\addbibresource{references.bib}
\externaldocument[E-]{\subfix{thesis}}

\abstract{%
    TODO
}

\acknowledgements{%
  TODO
}

\usepackage{geometry}
\setlength{\marginparwidth}{3cm}
\geometry{margin=3.8cm}

\begin{document}

\frontmatter{}

\listoffigures

\chapter{Duration from biomarkers} \label{ATACCC}

\section{Producing duration from parameter estimates}

The ATACCC analysis produces a posterior distribution for some population-level parameters, $\phi$.
Taking a posterior sample for this, $\phi^{(i)}$, we can then draw the parameters which determine the viral
load trajectory for a random individual $j$, $\theta_j$, from which that individual's duration $d_j^{(i)}$ can be calculated.
By drawing $\theta_1, \theta_2, \dots, \theta_n$ from the distribution $p(\theta_j \mid\phi^{(i)})$, the density of the duration distribution, $f_A(t)$, can be
calculated for the posterior sample $i$ as:
$$
f^{(i)}_A(t) = \frac{1}{n}\sum_{j=1}^n \mathbb{I} \left( t-0.5 < d_j^{(i)} < t+0.5 \right)
$$
Given $f_A(t)$, the cumulative density and hazard functions can be calculated.
Repeating this for $N$ posterior samples of $\phi$ gives the posterior distribution.
I have used $N = 1000$ and $n = 100,000$, 
Since implementing this solution, I have found that Monte Carlo integration to calculate durations given $\theta_i$ is more accurate by avoiding discretisation errors.
However, the results here do not use this for consistency.

\subfile{cis-perfect-testing}

\subfile{cis-imperfect-testing}

\appendix

\subfile{appendix-cis-perfect-testing}

\chapter{CIS episode definitions} \label{episode-def}

\todo[inline]{Get this from SARAH}

\chapter{Text to use}

Random bits of text that I've written that don't currently fit anywhere but will likely be useful in future

\section{CIS study description}
The CIS continually enrols individuals and tests them on an ongoing basis, following a
specified testing protocol. The protocol specifies that the initial five
tests are spaced every 7 days, then the spacing is every 28 days.
However, real-world considerations often lead to variations in test
times and missed tests. Our analysis focuses solely on the binary result
(positive or negative) of these tests, assuming no misclassification
bias.

The CIS (Coronavirus Infection Survey) is a longitudinal study on a representative sample of households.
A full description of the study can be found in the supplementary materials of \textcite{pouwelsCommunity} or the study protocol~\cref{cisProtocol}.
In brief, households are invited to the study from databases held by the ONS.
The number invited was to satisfy various target number of individuals swabbing per fortnight; relevant to the discussion in this document is that the target increased from 2,500 by the end of May 2020 to 150,000 in October 2020.
Within a household, all individuals aged 2 and over were invited to participate.
Once invited, an enrolment swab would be taken followed by 4 further weekly swabs (giving a total of 5 swabs on days 0, 7, 14, 21, 28 relative to enrolment) after which monthly swabs are taken.

During the period we consider, the total cohort size expands due to the continuous recruitment into the study.
Following this, the number then decreases and stabilises as those recruited to meet the October target transition from weekly onto monthly testing.
See figure \@ref(fig:test-schedules) for further detail.

\todo[inline]{Add some descriptive figures e.g. those in \url{~/COVID/ons-incidence/duration_estimation/reproducible-sims/reports/simulation-studies.pdf}}

\end{document}