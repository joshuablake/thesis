\documentclass[thesis.tex]{subfiles}

\title{Backcalculation}
\author{Joshua Blake}
\date{\today}

\begin{document}

\ifSubfilesClassLoaded{
  \setcounter{chapter}{6}
}

\chapter{Estimating incidence using backcalculation} \label{backcalc}

\todo[inline]{
    There is an asymmetry with this chapter and the following one on SEIR modelling.
    In this chapter, the theory/background is explained in an introductory/background chapter, with a short chapter with the application-specifics and results here.
    In the following chapter, the theory/background is explained within the chapter which means it is much longer.
    I'm not sure if this is an issue? But this chapter seems quite short currently and the next one quite long.
    I could move the SEIR theory to the background chapter, and possibly combine this chapter with the next one into a single chapter on incidence and transmission estimation?
    This would make sense for allowing a comparison between the results into its own section as well, rather than currently it being placed slightly awkwardly in the results and discussion sections of the SEIR chapter.
}


Now that I have estimated the duration of PCR positivity, I can estimate the incidence of SARS-CoV-2.
In this chapter, I will apply the theory introduced in \cref{E-incidence-prev} to the CIS data.
I use the period from the expansion of the CIS, starting 31st August 2020, until there was significant vaccination rollout\todo{insert date}.
First, I explain the method I will apply (\cref{backcalc:sec:methods}) and then present the results (\cref{backcalc:sec:results}).
I discuss the results in \cref{backcalc:sec:discussion} and conclude in \cref{backcalc:sec:conclustion}.

\todo[inline]{
    I should discuss the literature on estimating incidence based on SARS-CoV-2 prevalence data.
    In particular: \textcites{mccabeCISincidence}{abbottCISincidence}.
    However, I think this goes better in the introduction, framing the whole thesis because it is important in understanding the motivation for the whole work.
    At the same place, I could review the literature on mechanistic models using REACT and CIS.
}

\section{Methods} \label{backcalc:sec:methods}

As discussed in\todo{ref section}, a major issue with the CIS is non-response bias.
However, a deterministic relation between true prevalence and incidence is a good approximation (see \cref{E-incidence-prev:sec:prevalence-process}).
I extend the methodology previously used to estimate prevalence from the CIS data, corrected for non-response bias, to estimate incidence\todo{cite Koen's stuff}.
The previous method was multilevel regression and poststratification (MRP)\todo{cite MRP}.
I fit a similar model to estimate prevalence, then transform the posterior distribution of prevalence in each stratum to a posterior distribution of incidence.
Finally, I poststratify the posterior distribution of incidence to the CIS target population and subpopulations of interest.

The advantage of fitting a multilevel regression model on prevalence, rather than incidence, is that prevalence can naturally be expressed as a generalised linear model (GLM).
A natural formulation for either incidence proportion or prevalence in a stratum $i$ is $\logit(\pi_i) = g(t)$, where $\pi_i$ is the incidence proportion or prevalence in stratum $i$ and $g$ is a function of time, penalised towards being linear.
This is a GLM with a logit link function.
The logit transformation serves two purposes.
First, it ensures that the incidence proportion or prevalence is between 0 and 1.
Second, at low values of incidence or prevalence, linear growth is approximately exponential which is the natural assumption for an epidemic and has been shown to improve estimates of prevalence\todo{cite REACT paper on this, and check the exact meaning of "improve"}.
However, if the incidence proportion is expressed as a GLM in this way, the prevalence is not a GLM because a non-linear transformation is required to obtain prevalence from the logit of incidence proportion (the relationship on the natural scale is linear, as derived in \cref{E-incidence-prev:sec:prevalence-process}).
Using a GLM is computationally useful due to the large number of pre-existing packages available for computing the posterior distribution, or its approximation, efficiently.
Previously, when fitting to prevalence over long periods and including all appropriate covariates to correct for non-response bias, the model was too computationally expensive to fit using MCMC; instead, an implementation of the model using the Integrated Nested Laplace Approximation (INLA) algorithm was used~\citePersonalComms{Koen Pouwels}.

I modified a pre-existing version of a MRP model for CIS~\autocite{pouwelsMRPvaccination}, as recommended by that model's author for this situation~\citePersonalComms{Koen Pouwels}.
The model included fixed effects for age group\todo{check the model I used}.
The likelihood is a beta-binomial distribution, a generalisation of the binomial distribution allowing for overdispersion (see \cref{E-distributions}).
The overdispersion allows for household clustering\todo{ref beta-binomial as a model for clustering} and any other overdispersion in the data\todo{ref that epi data is often overdispersed}. 
Finally, the model was fit independently for each region, implicitly adding interactions between region and all other variables.
This step assumes that the estimates for each region can be considered independent, which is reasonable because the regions are large with many tests conducted within each.

I used age groupings:\todo{insert age groupings}.
These align with previous work on the CIS\todo{cite CIS reports}.
They are based on school ages (primary, compulsory secondary, and post-compulsory secondary) for children to capture that school opening and closing may have a large effect on transmission.

Having fit the prevalence model, I draw 10,000\todo{check number} prevalence trajectories in each stratum from the approximate posterior distribution of the model.
For each stratum, I use the relationship in \cref{E-inc-prev:eq:EPt} to calculate the incidence proportion at each time point.
Considering the relationship between incidence and prevalence to be deterministic, as \cref{E-inc-prev:sec:prevalence-process} justifies, equation implies that, for any stratum $\vec{P} = \matr{S} \vec{Z}$ where $\vec{P}$ is a vector of the number of prevalent individuals at each time point in the stratum, $\matr{S}$ is a matrix as defined below, and $\vec{Z}$ is a vector of the incidence at each time point in the stratum.
The matrix $\matr{S}$ needs to be defined based on the survival function to preserve the relationship between incidence and prevalence.
It is lower-triangular of the following form:
\begin{align}
    (\matr{S})_{i,j} = \begin{cases}
        S(i - j + 1) & i - \dmax - 1 \leq j \leq i \\
        0 & \text{otherwise.}
    \end{cases}
\end{align}
Since the matrix is lower-triangular, the solution for $\vec{P}$ can be found efficiently using forward substitution\todo{cite forward substitution}.
Forward substitution is numerically stable unless the diagonal elements are close to 0.
Here, the diagonal is $S(1) = 1$ and hence the matrix is well-conditioned.
\todo{I explained this more clearly in my first-year report, look that up!}

The final step is poststratifying the incidence proportion to the CIS target population and subpopulations of interest.
For each posterior draw and day\todo{introduce discrete time in days earlier}, the incidence proportion in the (sub)population is the weighted average of the incidence proportion in each stratum, where the weights are the proportion of the population in each stratum.
The population numbers for each stratum are provided by the Office for National Statistics.

\section{Results} \label{backcalc:sec:results}

\section{Discussion} \label{backcalc:sec:discussion}

\section{Conclusion} \label{backcalc:sec:conclustion}


\ifSubfilesClassLoaded{
  \listoftodos
}{}

\end{document}