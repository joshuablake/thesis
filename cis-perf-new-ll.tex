\documentclass[thesis.tex]{subfiles}

\begin{document}

\chapter{Alternative derivation of \cref{E-perf-test:eq:multinomial}} \label{perf-test:sec:alt-likelihood}

\section{Attempt 1}

Denote the characteristics of episode $j$ as $\nu_j = (B_j, E_j, i(j))$.
I now define a series of sets $\set{E}_1, \dots, \set{E}_{N_E}, \set{E}_u$ and an integer $N_E$ such that any possible $\nu_j$ is in exactly one of the classes and each set corresponds to the possible infection episdoes that lead to identical observations (conditional on the test schedules).

Define $\set{E}_u$ such that $\nu_h \in \set{E}_u$ if and only if $\nu_j$ would be undetected (if it occurred).
Hence, $\prob(\nu_j \in \set{E}_u \mid \vec{\theta}) = p_u$.

Otherwise, the sets are defined such that $\nu_j$ and $\nu_{j'}$ are in the same class if and only if $i(j) = i(j')$ and the episodes beginning and end times are censored to the same interval.
If $i \in \set{D}$ then there is exactly one $k$ such that $\set{E}_k$ is identical to $i$'s admissible region $\alpha_i$.
% In this case, I write $\alpha_{i(k)}$ to denote this admissible region. 

More formally, the sets can be defined in terms of an equivalence relation $\equiv$ and two arbitrary possible episodes $j$ and $j'$.
Define $\nu_j \equiv \nu_{j'}$ to hold if and only if either both $j$ and $j'$ would be undetected (as defined in \cref{E-perf-test:sec:prob-undetected}) or the following four conditions all hold.
\begin{enumerate}
    \item Both $\nu_j$ and $\nu_{j'}$ would be detected.
    \item $i(j) = i(j')$.
    \item $\tau_{\sched_{i(j)}}(B_j) = \tau_{\sched_{i(j')}}(B_{j'})$, where $\tau$ is as defined in \cref{perf-test:eq:tau-def}.
    \item $\tau_{\sched_{i(j)}}(E_j) = \tau_{\sched_{i(j')}}(E_{j'})$.
\end{enumerate}
Then each $\set{E}_k$ is an equivalence class under this relation and the set of all possible $\nu_j$.
That is, taking any $k = 1, \dots, N_E, u$ then $\nu_j \in \set{E}_k$ and $\nu_{j'} \in \set{E}_k$ if and only if $\nu_j \equiv \nu_{j'}$.

Let $\xi_k$ denote $\prob(\nu_j \in \set{E}_k)$.
Assume, for tractability, that the events $\nu_j \in \set{E}_k$ and $\nu_{j'} \in \set{E}_{k'}$ are independent for $j \neq j'$.
Denote by $m_k$ then number of episodes in $\set{E}_k$.
$\vec{m} = [m_1, \dots m_{N_E}, m_u]^T$ are the observations.
Then:
\begin{align}
  \vec{m}
  \mid \ntot, \vec{\theta}
  \dist
  \MN\left(
    \ntot,
    \frac{1}{\Ncis}
    \begin{bmatrix}
        \xi_1 \\ \xi_2 \\ \vdots \\ \xi_{N_E} \\ \xi_u
    \end{bmatrix}
  \right).
  \label{perf-test:eq:alt-likelihood}
\end{align}

If the set $\set{E}_k$ aligns with some $\alpha_i$ then $m_k = 1$ and $\xi_k = p_{ia}$.
Otherwise, for $k \neq u$, $m_k = 0$.
Finally, $m_u = n_u$ and $\xi_u = p_u$.
Therefore, the probability masses of \cref{perf-test:eq:alt-likelihood} and \cref{E-perf-test:eq:multinomial} match.



\section{Attempt 2}

\subsection{Perfect testing}

The test schedule $\sched_i$ is the set of times individual $i$ is tested at, starting from their last test prior to time 1, if it exists, or their time of enrolment otherwise.
Note that $\sched_i$ includes any tests that occur after $T$.

For a detected episode $j$, we observe $O_j = [l_j^{(b)}, r_j^{(b)}, l_j^{(e)}, r_j^{e}, i(j)]^T$.
These are: the individual in which the episode occurs, denoted $i(j)$; the earliest and latest time the episode could have begun, $l_j^{(b)}$ and $r_j^{(b)}$ respectively; and the earliest and latest time the episode could have ended, $l_j^{(e)}$ and $r_j^{(e)}$ respectively.

For an undetected episode $j$, there are no observations.
In which case, define $O(j) = \emptyset$.

Define an integer $D$ and $\set{D} = \vec{\nu}_1, \dots, \vec{\nu}_D$ as the set of all possible observations of detected episodes; that is, $O(j) \in \set{D}$ if and only if $j$ is a detected infection.
For any $\vec{\nu} = [l^{(b)}, r^{(b)}, l^{(e)}, r^{e}, i]^T$, $\vec{\nu} \in \set{D}$ if and only if the following conditions hold.
\begin{enumerate}
  \item $l^{(b)} - 1, r^{(b)}, l^{(e)}, r^{e} + 1 \in \sched_i$, that is individual $i$ was tested at all these times.
  \item $r^{(b)} = \min\{t \in \sched_i \ssep t \geq l^{(b)}\}$, that is $r^{(b)}$ is the first time at or after $l^{(b)}$ that individual $i$ was tested.
  \item $r^{(e)} = \min\{t \in \sched_i \ssep t > l^{(e)}\} - 1$, that is $r^{(e)}$ is the day before the next time after $l^{(e)}$ that individual $i$ was tested.
  \item $1 \leq r^{(b)} \leq T$, that is the episode begins in the period of interest.
\end{enumerate}
Let $n_k$ denote the number of observations of $\vec{\nu}_k$, $\nnodet$ the latent number of undetected episodes, $\vec{n} = [n_1, \dots, n_D, \nnodet]^T$, and $\ntot = \sum_{i=1}^D n_i + \nnodet$.
For $\vec{\nu}_k \in \set{D}$, let $p_k = \prob(O(j) = \vec{\nu}_k \mid \vec{\theta})$, $p_u = \prob(O(j) = \emptyset \mid \vec{\theta})$, and $\vec{p} = [p_1, \dots, p_D, p_u]^T$.
Assume, for tractability, that the events $O(j) = \vec{\nu}_k$ and $O(j') = \vec{\nu}_{k'}$ are independent for $j \neq j'$; this assumption is discussed in \cref{perf-test:sec:discussion}.
Then:
\begin{align}
  \vec{n} \mid \ntot, \vec{\theta} &\dist \MN(\ntot, \vec{p})
\intertext{that is:}
  p(\vec{n} \mid \ntot, \vec{\theta}) &= \frac{\ntot!}{\nnodet!\prod_{k=1}^D n_k!} p_u^{\nnodet} \prod_{k=1}^D p_k^{n_k}.
  \label{perf-test:eq:multinomial-ll}
\end{align}

In the CIS data, each $n_k$ is either 0 or 1.
Let $k \in \set{C}$ indicate that $n_k = 1$.
Furthermore, note that the support of the multinomial distribution requires that $\nnodet = \ntot - \ndet$.
Then \cref{perf-test:eq:multinomial-ll} simplifies to:
\begin{align}
  p(\vec{n} \mid \ntot, \vec{\theta})
  &= \frac{\ntot!}{(\ntot - \ndet)!} p_u^{\ntot-\ndet} \prod_{k \in \set{C}} p_k.
  \label{perf-test:eq:multinomial}
\end{align}

This is the previous likelihood where each $p_{ia}$ corresponds to a $p_k$ and $p_u$ is unchanged.

\subsection{Imperfect testing}

Just a sketch, probably ignore for now.

\begin{itemize}
  \item Add $\vec{y}_j$ as the observed test results for episode $j$ to $O(j)$ (as defined previously).
  \item Modify $p_k$ and $p_u$ as done currently.
\end{itemize}

\end{document}