\documentclass{cam-thesis}
\usepackage{setspace}
\onehalfspacing
\usepackage{graphicx}
\usepackage[utf8]{inputenc}

\usepackage{amsmath}
\usepackage{amssymb}
\usepackage{floatpag}
\usepackage{bm}
\usepackage[nameinlink]{cleveref}
\usepackage[style=british]{csquotes}
\usepackage[T1]{fontenc}
\usepackage{hyperref}
\usepackage{textcomp} % provide symbols
\usepackage{todonotes}
\usepackage{xr}
% \usepackage{microtype}

% Macros for common abbreviations to get the spacing right
% See: https://stackoverflow.com/questions/3282319/correct-way-to-define-macros-etc-ie-in-latex
\usepackage{xspace}
\makeatletter
\DeclareRobustCommand\onedot{\futurelet\@let@token\@onedot}
\def\@onedot{\ifx\@let@token.\else.\null\fi\xspace}
\def\eg{e.g\onedot} \def\Eg{\emph{E.g}\onedot}
\def\ie{i.e\onedot} \def\Ie{\emph{I.e}\onedot}
\def\cf{c.f\onedot} \def\Cf{\emph{C.f}\onedot}
\def\etc{etc\onedot} \def\vs{\emph{vs}\onedot}
\def\wrt{w.r.t\onedot} \def\dof{d.o.f\onedot}
\def\etal{et al\onedot}
\makeatother

% Generic maths commands
\def\reals{\mathbb{R}}
\def\nats{\mathbb{N}}
\def\sampSpace{\mathcal{X}}
\def\dist{\sim}
\DeclareMathOperator{\E}{\mathbb{E}}
\DeclareMathOperator{\V}{\mathbb{V}}
\DeclareMathOperator{\prob}{\mathbb{P}}
\DeclareMathOperator{\p}{\pi}
\DeclareMathOperator{\var}{\mathbb{V}}
\DeclareMathOperator{\indicator}{\mathbb{I}}
\DeclareMathOperator{\cov}{Cov}
\DeclareMathOperator{\logit}{logit}
\DeclareMathOperator{\Ber}{Bernoulli}
\DeclareMathOperator{\Bin}{Binomial}
\DeclareMathOperator{\Poi}{Poisson}
\DeclareMathOperator{\Beta}{Beta}
\DeclareMathOperator{\Exponential}{Exponential}
\DeclareMathOperator{\NegBin}{NegBin}
\DeclareMathOperator{\GamDist}{Gamma}
\DeclareMathOperator{\MN}{Multinomial}
\newcommand\matr{\bm}
\newcommand\set{\mathcal}
\renewcommand{\vec}{\bm}
\newcommand{\ssep}{:}

% Thesis-specific maths commands
\newcommand{\psens}{p_\text{sens}}
\newcommand{\ntot}{n_\text{tot}}
\newcommand{\ndet}{n_\text{d}}
\newcommand{\nnodet}{n_\text{u}}
\newcommand{\pnodet}{p_\text{u}}
\newcommand{\Ncis}{N_\text{CIS}}
\newcommand{\ncis}{\vec{n_\text{CIS}}}
\newcommand{\na}{\vec{n_\text{a}}}
\newcommand{\pcis}{\vec{p_\text{CIS}}}
\newcommand{\sched}{\mathbb{T}}

\newcommand\citePersonalComms[1]{(#1, personal communication)}


\title{Estimating SARS-CoV-2 transmission in England from randomly sampled prevalence surveys}
\author{Joshua Blake}
\date{\today}
\subjectline{MRC Biostatistics Unit}
\college{Jesus College}
\collegeshield{CollegeShields/jesus}
% \keywords{one two three}

%% Submission date [optional]:
% \submissiondate{November, 2042}

%% You can redefine the submission notice [optional]:
% \submissionnotice{A badass thesis submitted on time for the Degree of PhD}

% Only number labelled equations
% \forlistloop workaround due to incompatability with biblatex
% See: https://tex.stackexchange.com/questions/220268/biblatex-and-autonum-dont-work-together
\usepackage{etoolbox} % get the good meaning of \forlistloop
\let\etoolboxforlistloop\forlistloop % save the good meaning of \forlistloop
\usepackage{autonum}
\let\forlistloop\etoolboxforlistloop % restore the good meaning of \forlistloop
\makeatletter
\let\blx@noerroretextools\empty
\makeatother
\usepackage[backend=biber,style=authoryear,uniquename=false]{biblatex}
\addbibresource{references.bib}

\usepackage{subfiles} % Best loaded last in the preamble
\addbibresource{references.bib}
\externaldocument[E-]{\subfix{thesis}}

\abstract{%
    TODO
}

\acknowledgements{%
  TODO
}

\usepackage{geometry}
\setlength{\marginparwidth}{3cm}
\geometry{margin=3.8cm}

\begin{document}

\frontmatter{}

\listoffigures

\chapter{Introduction} \label{intro}

\section{The COVID-19 pandemic and ongoing burden}

\section{SARS-CoV-2 and COVID-19}

\section{Thesis outline}

\chapter{Incidence, prevalence, and duration}
We wish to estimate the distribution of time for which an individual is positive for.
An individual could have multiple infections, and hence I refer to \emph{infection episodes}.

I define an individual as \emph{detectable} if they would test positive if tested, in the absence of misclassification error (which is dealt with in \cref{E-imperf-test}).
Episode $i$'s \emph{duration of positivity}, $D_i$ is the number of days that they are detectable for; I use time discretised to days, therefore, the duration is at least 1 day.
Note that the relationship between positivity (\ie the result of a PCR test) and infectiousness (\ie an individual's ability to pass on the virus) is complex~\autocites{lascolaViral}{singanayagamDuration}, and beyond the scope of this thesis.
I define an episode as starting at time $B_i$ and ends at $E_i$, hence $D_i = E_i - B_i + 1$.

An episode is \emph{detected} when they test positive at least once, if this occurs; episode $i$ is detected at time $t$ if and only if: $i$ is detectable at $t$, $i$ is tested at $t$, and the test returns positive (a false negative does not occur).

The survival time function, parameterised by $\theta$, is the probability that an episode lasts at least $t$ days: $S_\theta(t) = \prob(D_i \geq t \mid B_i = b, \theta)$, where $D_i$ and $B_i$ are the duration and starting time of episode $i$ respectively.
I make the standard assumption that $D_i$ and $B_i$ are independent and hence: $S_\theta(t) = \prob(D_i \geq t \mid \theta)$.I utilise survival analysis for analysing this data.
Survival analysis is the area of statistics concerned with estimating the distribution of the times between events.
It is broadly applicable to many domains, such as the time-to-failure for a mechanical system or the effect of a treatment on time spent in hospital.
In the context of this thesis, the distribution to estimate is that of duration of positivity.
The \emph{initiating event} is when the duration starts, \ie becoming detectable.
The \emph{terminating event} is when the duration ends, \ie becoming no longer detectable.

\chapter{Studies used in this thesis} \label{intro:sec:studies}

\section{PCR testing} \label{intro:sec:PCR}

\section{Assessment of Transmission and Contagiousness of COVID-19 in Contacts}

\section{Coronavirus (COVID-19) Infection Survey} \label{intro:sec:cis}

The CIS (Coronavirus (COVID-19) Infection Survey) continually enrols individuals and tests them on an ongoing basis, following a
specified testing protocol. The protocol specifies that the initial five
tests are spaced every 7 days, then the spacing is every 28 days.
However, real-world considerations often lead to variations in test
times and missed tests. Our analysis focuses solely on the binary result
(positive or negative) of these tests, assuming no misclassification
bias.

The CIS (Coronavirus Infection Survey) is a longitudinal study on a representative sample of households.
A full description of the study can be found in the supplementary materials of \textcite{pouwelsCommunity} or the study protocol~\autocite{cisProtocol}.
In brief, households are invited to the study from databases held by the ONS.
The number invited was to satisfy various target number of individuals swabbing per fortnight; relevant to the discussion in this document is that the target increased from 2,500 by the end of May 2020 to 150,000 in October 2020.
Within a household, all individuals aged 2 and over were invited to participate.
Once invited, an enrolment swab would be taken followed by 4 further weekly swabs (giving a total of 5 swabs on days 0, 7, 14, 21, 28 relative to enrolment) after which monthly swabs are taken.

During the period we consider, the total cohort size expands due to the continuous recruitment into the study.
Following this, the number then decreases and stabilises as those recruited to meet the October target transition from weekly onto monthly testing.
See figure \@ref(fig:test-schedules) for further detail.

\todo[inline]{Add some descriptive figures e.g. those in \url{~/COVID/ons-incidence/duration_estimation/reproducible-sims/reports/simulation-studies.pdf}}

Consider the CIS data.
It consists of a series of tests on the same individuals (longitudinal data), where some individuals have a series of positive tests.

\chapter{Duration from biomarkers} \label{ATACCC}

\section{Background}

\section{Original analysis}

\section{Comparison to \texorpdfstring{\textcite{hakkiOnset}}{Hakki \etal (2022)}}

\section{Final analysis}

\section{Producing duration from parameter estimates}

The ATACCC analysis produces a posterior distribution for some population-level parameters, $\phi$.
Taking a posterior sample for this, $\phi^{(i)}$, we can then draw the parameters which determine the viral
load trajectory for a random individual $j$, $\theta_j$, from which that individual's duration $d_j^{(i)}$ can be calculated.
By drawing $\theta_1, \theta_2, \dots, \theta_n$ from the distribution $p(\theta_j \mid\phi^{(i)})$, the density of the duration distribution, $f_A(t)$, can be
calculated for the posterior sample $i$ as:
$$
f^{(i)}_A(t) = \frac{1}{n}\sum_{j=1}^n \mathbb{I} \left( t-0.5 < d_j^{(i)} < t+0.5 \right)
$$
Given $f_A(t)$, the cumulative density and hazard functions can be calculated.
Repeating this for $N$ posterior samples of $\phi$ gives the posterior distribution.
I have used $N = 1000$ and $n = 100,000$, 
Since implementing this solution, I have found that Monte Carlo integration to calculate durations given $\theta_i$ is more accurate by avoiding discretisation errors.
However, the results here do not use this for consistency.

\subfile{cis-perfect-testing}

\subfile{cis-imperfect-testing}

\chapter{Statistical backcalculation}

\subfile{SEIR}

\chapter{Conclusion}

\listoftodos

\printbibliography

\appendix

\chapter{CIS episode definitions} \label{episode-def}

\todo[inline]{Get this from SARAH}

\end{document}