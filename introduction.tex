\documentclass[thesis.tex]{subfiles}

\begin{document}

\chapter{Introduction} \label{intro}

\begin{itemize}
    \item Disease burden of pandemic and ongoing burden of COVID
    \item To mitigate its effects, need to understand incidence and transmission but cannot observe directly
    \item Importance of surveillance objectives and concepts: incidence, transmission, basic reproduction number, effective reproduction number
    \item Expected future pandemics, methods should be reusable
\end{itemize}

- Natural history of SARS-CoV-2 infection
	- Stylised graphic of phases: infection, pre-infection positive, infectious positive, post-infectious positive, negative
	- How it spreads
- Why is incidence and prevalence estimation challenging?
	- Asymptomatic infection
	- Ascertainment bias
- How do we estimate prevalence
	- CIS and REACT
- Issues with current incidence estimation

\section{SARS-CoV-2 real-time surveillance}

\begin{itemize}
    \item Most commonly used types of data: severe events (deaths/hospitalisations), widespread testing (pillar 2), 
    \item Advantages of designed surveys (REACT, CIS)
    \item Use of cohorts (ATACCC)
    \item Statistical modelling performed during the pandemic. Discuss broadly but especially focus on those using CIS or REACT. Discuss my contribution in this context.
\end{itemize}


\section{Historical context}

\begin{itemize}
    \item Infectious disease epidemiology and biostatistics heavily developed in context of HIV/AIDS epidemic
    \item Use much methodology and concepts from this field
\end{itemize}

\subsection{Backcalculation}

\emph{Backcalculation} is the procedure for estimating incidence based on a measurement of prevalence.
\Textcite{brookmeyerMethod} developed it in the context of HIV (human immunodeficiency virus) and AIDS (acquired immune deficiency syndrome).
HIV is transmitted sexually or through blood and causes the disease AIDS.
It has killed 40 million people, peaking at 2 million annual deaths~\autocite{unaids2023}.
In this context, backcalculation estimates the incidence of HIV infection from: data on the number of AIDS diagnoses, and the incubation distribution (the time between infection and AIDS diagnosis)~\autocites{brookmeyerBackcalculation}{brookmeyerMeasuring}.
AIDS is incurable, therefore, AIDS prevalence is equivalent to cumulative incidence.


The original formulations of backcalculation neglected the observation process.
AIDS is almost always diagnosed and recorded in developed countries.
That is, the observations of AIDS diagnoses are the same as the true incidence of AIDS.
An exception is that the reporting delay may be important~\autocite{paganoHIV}.
Reporting delay is not relevant to the studies in this thesis, and I ignore it in this and further discussion.
Therefore, two processes needed to be considered in the context of HIV/AIDS: infection and prevalence.

For SARS-CoV-2, measurements of prevalence are noisy because they are based on population sampling.
The sampling is described as the observation process.
The observation process noise is much larger than the noise of the prevalence process (see \cref{E-inc-prev:sec:observation-process}).
Therefore, the latter is negligible and again two processes need to be considered: infection and observation.

\begin{itemize}
    \item Duration is important
    \item Mathematical details in \cref{E-inc-prev}
\end{itemize}

\subsection{Mechanistic models}

In this chapter, I introduce mechanistic models for estimating the transmission of SARS-CoV-2, an alternative to the statistical approach in \cref{E-backcalc}.
Mechanistic models have a structure which make assumptions on the biological and epidemiological mechanisms behind the infection process (see \cref{E-inc-prev:sec:infection-process}), and hence the disease dynamics~\autocite{lesslerMechanistic}.
Individuals are giving the model's state and parameters a biological and/or epidemiological interpretation.
% Of particular interest is that the basic reproduction number, $\R$, and the effective reproduction number, $R_e(t)$, which can both be calculated as functions of the model parameters; using \cref{E-backcalc}'s approach, additional modelling would be required to estimate these numbers.
In particular, this allows changes in transmission to be decomposed into the contributing factors, \eg the number of contacts per day or the probability of transmission upon contact.
The ability to estimate the contribution of each of these factors can inform public health policy.
Furthermore, mechanistic models can be used for scenario-based modelling, which simulates the effect of interventions.
A scenario here is the effect of a proposed intervention (or lack thereof), introduced into the model by modifying the relevant parameters.
The mechanistic model can then be simulated forward in time to understand the intervention's effects.
All of these properties make mechanistic models a useful tool for understanding and controlling infectious diseases.
However, they make stronger assumptions about the disease and population behaviour than statistical models.

The literature on mechanistic models is vast, having been developed for over a century.
The model in \cref{E-SEIR:sec:SIR} was first formulated by \textcite{kermackContribution}.
However, the earliest mechanistic model to be formulated mathematically was probably \textcite{rossMalariaA}'s model of malaria transmission~\autocite{lesslerMechanistic}.
These early contributions led to important insights for public health.
\Textcite{rossMalariaA} showed that, if mosquitoes were controlled but not eliminated, malaria could be controlled.
\Textcite{kermackContribution} explained the observation that epidemics ended before all individuals were infected due to a population-level build-up of immunity known as \emph{herd immunity}.

A full review of this literature is beyond the scope of this thesis.
In this chapter, I will discuss only the background relevant to the application.
For further details I recommend the tutorial paper \textcite{kretzschmarMathematical} or the textbook \textcite{keelingModeling}.

I start this chapter by introducing a generic framework for mechanistic modelling of infectious diseases (\cref{E-SEIR:sec:transmission-generic}).
% This framework is then applied to SARS-CoV-2 in \cref{E-SEIR:sec:transmission-application}.
It describes the \emph{transmission model}, corresponding to the infection process.
The transmission model needs to be linked to observations; in this chapter, that will be the CIS prevalence data.
The \emph{observation model}, introduced in \cref{E-SEIR:sec:observation}, is this link; it incorporates both the prevalence process and the observation process.
These two components form the mathematical model, which I then fit to data, requiring specification of an inference process, parameterisation, and priors (\cref{E-SEIR:sec:inference-implementation}).
This final formulation uses the transmission model from \textcite{birrellRealtime}, but a novel observation model.
I then use a simulation study to check this recovers the parameters successfully (\cref{E-SEIR:sec:sim-study}).
Having shown that the model can recover the parameters, I then apply it to the CIS data (\cref{E-SEIR:sec:application}).
Finally, I discuss the results, limitations, and extensions of this work (\cref{E-SEIR:sec:discussion}).

Notation in this chapter differs from the rest of this thesis.
I adopt conventions and notation familiar to the infectious disease modelling field, although the field's notation is not consistent between sources.
In particular, upper or lower case no longer signifies whether a variable is random or not, but this is clear from the context.

\subsection{Duration} \label{inc-prev:sec:duration}

Two approaches are commonly used to estimate durations: either modelling an underlying biomarker or modelling the duration directly~\autocite{sweetingEstimating}.
For SARS-CoV-2, both approaches would be based on the results of PCR testing.
Modelling a biomarker would mean modelling the viral load, measured as a Ct value (see \cref{E-ATACCC}).
Modelling the duration would consider only the binary result at each test (positive or negative); in this thesis a \emph{survival analysis} framework is adopted (see \cref{E-perf-test,E-imperf-test}).
% Survival analysis is the area of statistics concerned with estimating the distribution of the times between two events.
Other related approaches, such as a multi-state model~\autocite{jacksonMSM}, could also be considered (multi-state models can be viewed as a generalisation of survival analysis).
Therefore, the biomarker approach considers more information per observation, however, a model for the biomarker is required which generally requires stronger assumptions.
Modelling a biomarker uses more information per observation, but requires stronger assumptions.

\section{Thesis contribution}

\begin{itemize}
    \item Overall question: how can prevalence surveys best be used to understand incidence and transmission?
    \item Inform future study design
    \item Provide methodology
\end{itemize}

The code for the analyses in this thesis is available online.
An index to the various repositories is available at\todo{insert GitHub link}.
Almost all code, excluding dependencies, was written from scratch for these analyses.
The exception is the Markov chain Monte Carlo sampler used in \cref{E-SEIR}, which was created by Sanmitra Ghosh for the article \textcite{ghoshApproximate}.

The ATACCC data used is found in the same location.
CIS data can only be accessed by accredited researchers within the Office for National Statistic's Secure Research Service\todo{insert details of applying}.
Where possible, aggregated, non-disclosive versions of the dataset are available alongside the code.

\subsection{Epidemiological contributions}

\begin{itemize}
    \item Estimate of duration of PCR-positivity: of interest for rest of thesis but also more broadly for public health (\eg interpreting PCR test results)
    \item Use in interpreting test results
    \item Incidence estimates from only CIS
    \item Transmission estimates from only CIS
\end{itemize}

\subsection{Methodological contributions}

\begin{itemize}
    \item Theory justifying the use of deterministic backcalculation in this setting. Already used in practice but without understanding when the approximation is good. In particular: all that is required is that the proportion of the population sampled is small, not that the prevalence or incidence is sufficiently large.
    \item Double interval censoring with arbitrary truncation and false negatives
    \item Combining ATACCC and CIS duration estimates
    \item SEIR model using only prevalence surveys
    \item Application within TRE
\end{itemize}

\subsection{Structure}

Outline the chapters and how they fit together

\end{document}