\documentclass[thesis.tex]{subfiles}

\begin{document}
\chapter{Tables of distributions} \label{distributions}

This appendix contains details of the distributions used in this thesis.

\begin{landscape}
\section{Continuous distributions}
\begin{tabular}{llp{3.5cm}lll}
Name & Notation & Parameters & Mean & Variance & pdf\\
\hline \\
Normal & $\N(\mu, \sigma^2)$ & $\mu$ (mean)\newline $\sigma^2 > 0$ (variance) & $\mu$ & $\sigma^2$ & $\frac{1}{\sqrt{2\pi\sigma^2}}e^{-\frac{(x-\mu)^2}{2\sigma^2}}$ \\
\makecell{Multivariate \\ normal} & $\MNorm(\vec{\mu}, \matr{\Sigma})$ & $\mu \in \reals^k$ (mean vector)\newline $\Sigma \in \reals^{k \times k}$ (covariance matrix, positive semi-definite) & $\vec{\mu}$ & $\matr{\Sigma}$ & $(2 \pi)^{-k / 2} \operatorname{det}(\matr{\Sigma})^{-1 / 2} \exp \left(-\frac{1}{2}(\vec{x}-\vec{\mu})^T \matr{\Sigma}^{-1}(\vec{x}-\vec{\mu})\right)$ \\
Lognormal & $\LN(\mu, \sigma^2)$ & $\mu$ (mean of the logarithm)\newline $\sigma^2 > 0$ (variance of the logarithm) & $e^{\mu + \frac{\sigma^2}{2}}$ & $(e^{\sigma^2} - 1)e^{2\mu + \sigma^2}$ & $\frac{1}{x\sigma\sqrt{2\pi}}e^{-\frac{(\ln(x)-\mu)^2}{2\sigma^2}}$ \\
Gamma & $\GamDist(\alpha, \beta)$ & $\alpha > 0$ (shape) \newline $\beta > 0$ (rate) & $\frac{\alpha}{\beta}$ & $\frac{\alpha}{\beta^2}$ & $\frac{\beta^\alpha}{\Gamma(\alpha)\theta}x^{\alpha-1}e^{-\beta x}$ \\
Beta & $\BetaDist(\alpha, \beta)$ & $\alpha > 0$ (shape)\newline $\beta > 0$ (shape) & $\frac{\alpha}{\alpha+\beta}$ & $\frac{\alpha\beta}{(\alpha+\beta)^2(\alpha+\beta+1)}$ & $\frac{1}{B(\alpha,\beta)}x^{\alpha-1}(1-x)^{\beta-1}$ \\
Exponential & $\Exponential(\lambda)$ & $\lambda$ (rate) & $\frac{1}{\lambda}$ & $\frac{1}{\lambda^2}$ & $\lambda e^{-\lambda x}$ \\
LKJ* & $\LKJ(\eta)$ & $\eta > 0$ (shape) & $\matr{I}$ (identity matrix) & N/A & $C \det(\matr{x})^{\eta-1}$ \\
\end{tabular}

* The LKJ distribution~\autocite{lewandowskiGenerating} has support over positive definite symmetric matrices $\matr{x}$.
These matrices are exactly the matrices that are valid correlation matrices.
The normalising constant, $C$, has a complex form and is rarely needed.

\section{Discrete distributions}
\begin{tabular}{llp{3.5cm}lll}
Name & Notation & Parameters & Mean & Variance & pmf \\
\hline \\
Bernoulli & $\Ber(p)$ & $0 \leq p \leq 1$ (probability of success) & $p$ & $p(1-p)$ & $p^x(1-p)^{1-x}$ \\
Binomial & $\Bin(n, p)$ & $n \in \nats$ (number of trials, integer)\newline $0 \leq p \leq 1$ (probability of success) & $np$ & $np(1-p)$ & ${n \choose x}p^x(1-p)^{n-x}$ \\
Poisson & $\Poi(\lambda)$ & $\lambda > 0$ (average rate) & $\lambda$ & $\lambda$ & $\frac{e^{-\lambda}\lambda^x}{x!}$ \\
Negative Binomial (standard) & $\NBs(r, p)$ & $r > 0$ (number of failures until stopping)\newline $0 < p < 1$ (probability of success) & $\frac{pr}{1-p}$ & $\frac{pr}{(1-p)^2}$ & $\frac{\Gamma(x+r)}{x! \Gamma(r)} p^x (1-p)^r$ \\
Negative Binomial (centred) & $\NBc(\mu, r)$ & $\mu > 0$ (mean) \newline r (dispersion) & $\frac{pr}{1-p}$ & $\frac{pr}{(1-p)^2}$ & $\frac{\Gamma(x+r)}{x! \Gamma(r)} \left( \frac{r}{r+\mu} \right)^\mu \left( \frac{\mu}{r+\mu} \right)^k$ \\
Beta-Binomial (centred) & $\BB(n, p, \rho)$ & $n \in \nats$ (number of trials, integer)\newline $0 \leq p \leq 1$ (success probability)\newline $\rho > 0$ (overdispersion) & $np$ & $np(1-p)(1+n\rho)(1+\rho)^{-1}$ & ${n \choose x} \frac{B(x+p/\rho, n-x+(1-p)/\rho)}{B(p/\rho, (1-p)/\rho)}$ \\
Multinomial & $\MN(n, \vec{p})$ & $n \in \nats$ (number of trials, integer)\newline $\vec{p} > 0$ (vector of $k$ probabilities, $\sum_{i=1}^k p_i = 1$) & $n \vec{p}$ & See below & $\frac{n!}{x_1!x_2!\ldots x_k!}p_1^{x_1}p_2^{x_2}\ldots p_k^{x_k}$ \\
\end{tabular}

$B(x,y)$ is the beta function, defined in terms of the Gamma function $\Gamma(x)$ as $B(x,y) = \frac{\Gamma(x)\Gamma(y)}{\Gamma(x+y)}$.
For $X \dist \MN(n, \vec{p})$, $\V(X_i) = np_i(1-p_i)$ and $\cov(X_i, X_j) = -np_ip_j$ for $i \neq j$.
\end{landscape}
\end{document}