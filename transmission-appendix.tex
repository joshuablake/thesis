\documentclass[thesis.tex]{subfiles}

\begin{document}

\chapter{Creating RT-PCR positive states} \label{transmission:sec:phase-type}

Deciding $l$ and the length of time each individual spends in each compartment is the most challenging part of formulating the observation model.
The class of distributions which can be represented as a series of compartments are known as \emph{phase-type} distributions.
Discussions of general phase-type distributions are beyond the scope of this thesis; a review in a biological context can be found in \textcite{hobolthPhasetype}.
I will restrict myself to the subset of phase-type distributions defined by \textcite{osogamiClosed} as Erlang-Coxian (EC) distributions.
EC distributions are convenient because they are flexible enough to approximate almost any distribution, and the parameters of the approximation are given by closed-form functions of the distribution's moments.

% Formally, a phase-type distribution as the distribution of \emph{absorption time} in a \emph{continuous-time Markov process} with a single absorbing state~\autocite{hobolthPhasetype}.
% Given a probability distribution over starting states, a rate at which 
% The process can be defined by its size, $l$, a $l \times l$ \emph{transition matrix}, $\matr{M} \in \reals^{l \times l}$, an \emph{exit rate} vector, $\vec{\alpha}$ 
% The process's state at time $t$, $W_t$, is a scalar $W_t \in \{ 1, \dots, l+1 \}$.
% Its transition matrix is defined, for $j \neq k$, as $\prob(W_{t+\delta t} = k \mid ) = M_{jk} \delta t$ for small, positive $\delta t$.
% The diagonal is defined as $M_{jj} = - \sum_{k \neq j} M_{jk}$ so that the matrix $\matr{M}$ has row sums 0.
% The state $l+1$ is an absorbing state, that is $\prob(W_{t+t'} = l+1 \mid W_t = l_1) = 1$ for any non-negative $t'$ (once the process enters it, it cannot leave).
% The exit rate gives the transition rates into the absorbing state: $\prob(W_{t+\delta t} = l + 1 \mid W_t = j) = \alpha_j \delta t$ for small $\delta t$.
% The process's absorption time is when it first arrives at the absorbing state, $\tau = \inf \{ t \ssep W_t = l+1 \}$.
% That is, $P$ is a phase-type distribution if it can be represented as a process 
% They are additions of arbitrary exponential distributions, and can (if there is no limit on the number of compartments used) represent any distribution.
% \todo{check/cite these claims}
% I use the method of \textcite{osogamiClosed} to determine $l$ and the transition rates between the compartments.

% A $l$-phase EC distribution is the sum of a $(l-2)$-phase gamma distribution and a two-phase acyclic, arbitrary phase-type distribution.
An $l$-phase EC distribution $P$ can be defined as the sum of three random variables: $K$, $X_1$, and $X_2$.
It is shown schematically in \cref{SEIR:fig:EC}.
Here, $K \dist \GamDist(l-2, \omega_Y)$, $X_1 \dist \Exponential(\omega_{X1})$ and $X_2$ is a mixture distribution taking the value 0 with probability $1-\pi$ or otherwise (\ie with probability $\pi$) distributed $\Exponential(\omega_{X2})$.
% The time between entering the 1st state and the $l+1$th state, the absorbing state, is the duration.
These distributions contain five parameters.
Three are positive reals, $\omega_Y$, $\omega_{X1}$, and $\omega_{X2}$; one is a probability, $\pi$, in the interval $[0, 1]$; and one is a positive integer, $l$.

The name Erlang-Coxian comes from the two distributions that make up the EC distribution.
Gamma distributions with integer first parameters (as $K$ has) are also known as Erlang distributions.
The distribution $X_1 + X_2$ is a special case of the Coxian distribution.
Hence, the overall distribution is known as a Erlang-Coxian distribution.

\begin{figure}
\makebox[\textwidth][c]{
\begin{tikzpicture}[
    node distance = 2.5cm,
    on grid,
    auto,
    ->,>=stealth',
    every state/.style={draw,rectangle},
    ]

    \node[state] (1) {1};
    \node[state, right of=1] (2) {2};
    \node[state, right of=2, draw=none] (dots) {$\dots$};
    \node[state, right of=dots] (lm2) {$l-2$};
    \node[state, right of=lm2] (lm1) {$l-1$};
    \node[state, right of=lm1] (l) {$l$};
    \node[state, right of=l, align=center] (absorbing) {absorbing\\state};

    \path (1) edge node {$\omega_Y$} (2)
          (2) edge node {$\omega_Y$} (dots)
          (dots) edge node {$\omega_Y$} (lm2)
          (lm2) edge node {$\omega_Y$} (lm1)
          (lm1) edge node {$\pi \omega_{X1}$} (l)
                edge [out=-45,in=-135] node[below] {$(1 - \pi) \omega_{X1}$} (absorbing)
          (l) edge node {$\omega_{X2}$} (absorbing);
\end{tikzpicture}
}
\caption[A $l$ phase EC distribution.]{Representation of an EC distribution as a progression through a series of compartments. The distribution is the time from entering state 1 and entering the absorbing state.}
\label{SEIR:fig:EC}
\end{figure}

\Textcite{osogamiClosed} define a phase-type distribution $P$ as \emph{well representing} an arbitrary distribution $G$ if $P$ and $G$ have the same first three moments.
They then prove that EC distributions have the following properties.
\begin{itemize}
    \item Let $\set{W}$ be the set of all phase-types distributions that well represent $G$.
        Let $\set{W}' \subseteq \set{W}$ be the set of EC distributions that well represent $G$.
        If $\set{W}$ is non-empty (which is true for most real-life distributions), then $\set{W'}$ is also non-empty; the following properties assume that $\set{W}$ is non-empty.
    \item Let $w$ and $w'$ be the minimum number of phases of any member of $\set{W}$ and $\set{W'}$ respectively.
        Then $w' = w$ or $w' = w + 1$.
        That is, the minimum number of phases for an EC distribution to well represent $G$ is at most one more than the minimum number of phases for any phase-type distribution to well represent $G$.
        Fewer phases is generally beneficial as it reduces the time and space complexity of working with the distribution.
    \item A $l$-phase EC distribution $P' \in \set{W'}$ with $l \leq w + 1$ can be found using a closed-form expression (given by \textcite{osogamiClosed}).
        This is beneficial because finding arbitrary members of $\set{W}$ is challenging.
\end{itemize}

A total of five parameters are required to specify a $l$-phase EC distribution.
\begin{itemize}
    \item $l$, the number of phases.
    \item $\omega_Y$, the rate parameter of the Erlang distribution.
    \item $\omega_{X1}$, the rate parameter of the first state after the Erlang distribution.
    \item $\omega_{X2}$, the rate parameter of the second state after the Erlang distribution.
    \item $\pi$, the probability of moving from the end state of the Erlang distribution to the second (as opposed to directly to the absorbing state).
\end{itemize}

\end{document}