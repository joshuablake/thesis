\documentclass[thesis.tex]{subfiles}

\begin{document}

\chapter{Introduction} \label{intro}

\begin{itemize}
    \item Disease burden of pandemic and ongoing burden of COVID
    \item To mitigate its effects, need to understand incidence and transmission but cannot observe directly
    \item Importance of surveillance objectives and concepts: incidence, transmission, basic reproduction number, effective reproduction number
    \item Expected future pandemics, methods should be reusable
\end{itemize}

- Natural history of SARS-CoV-2 infection
	- Stylised graphic of phases: infection, pre-infection positive, infectious positive, post-infectious positive, negative
	- How it spreads
- Why is incidence and prevalence estimation challenging?
	- Asymptomatic infection
	- Ascertainment bias
- How do we estimate prevalence
	- CIS and REACT
- Issues with current incidence estimation

\section{SARS-CoV-2 real-time surveillance}

\begin{itemize}
    \item Most commonly used types of data: severe events (deaths/hospitalisations), widespread testing (pillar 2), 
    \item Advantages of designed surveys (REACT, CIS)
    \item Use of cohorts (ATACCC)
    \item Statistical modelling performed during the pandemic. Discuss broadly but especially focus on those using CIS or REACT. Discuss my contribution in this context.
\end{itemize}


\section{Historical context}

\begin{itemize}
    \item Infectious disease epidemiology and biostatistics heavily developed in context of HIV/AIDS epidemic
    \item Use much methodology and concepts from this field
\end{itemize}

\subsection{Backcalculation}

\emph{Backcalculation} is the procedure for estimating incidence based on a measurement of prevalence.
\Textcite{brookmeyerMethod} developed it in the context of HIV (human immunodeficiency virus) and AIDS (acquired immune deficiency syndrome).
HIV is transmitted sexually or through blood and causes the disease AIDS.
It has killed 40 million people, peaking at 2 million annual deaths~\autocite{unaids2023}.
In this context, backcalculation estimates the incidence of HIV infection from: data on the number of AIDS diagnoses, and the incubation distribution (the time between infection and AIDS diagnosis)~\autocites{brookmeyerBackcalculation}{brookmeyerMeasuring}.
AIDS is incurable, therefore, AIDS prevalence is equivalent to cumulative incidence.


The original formulations of backcalculation neglected the observation process.
AIDS is almost always diagnosed and recorded in developed countries.
That is, the observations of AIDS diagnoses are the same as the true incidence of AIDS.
An exception is that the reporting delay may be important~\autocite{paganoHIV}.
Reporting delay is not relevant to the studies in this thesis, and I ignore it in this and further discussion.
Therefore, two processes needed to be considered in the context of HIV/AIDS: infection and prevalence.

For SARS-CoV-2, measurements of prevalence are noisy because they are based on population sampling.
The sampling is described as the observation process.
The observation process noise is much larger than the noise of the prevalence process (see \cref{inc-prev:sec:observation-process}).
Therefore, the latter is negligible and again two processes need to be considered: infection and observation.

\begin{itemize}
    \item Duration is important
    \item Mathematical details in \cref{E-inc-prev}
\end{itemize}

\subsection{Duration} \label{inc-prev:sec:duration}

Two approaches are commonly used to estimate durations: either modelling an underlying biomarker or modelling the duration directly~\autocite{sweetingEstimating}.
For SARS-CoV-2, both approaches would be based on the results of PCR testing.
Modelling a biomarker would mean modelling the viral load, measured as a Ct value (see \cref{E-ATACCC}).
Modelling the duration would consider only the binary result at each test (positive or negative); in this thesis a \emph{survival analysis} framework is adopted (see \cref{E-perf-test,E-imperf-test}).
% Survival analysis is the area of statistics concerned with estimating the distribution of the times between two events.
Other related approaches, such as a multi-state model~\autocite{jacksonMSM}, could also be considered (multi-state models can be viewed as a generalisation of survival analysis).
Therefore, the biomarker approach considers more information per observation, however, a model for the biomarker is required which generally requires stronger assumptions.
Modelling a biomarker uses more information per observation, but requires stronger assumptions.

\section{Thesis contribution}

\begin{itemize}
    \item Overall question: how can prevalence surveys best be used to understand incidence and transmission?
    \item Inform future study design
    \item Provide methodology
\end{itemize}

\subsection{Epidemiological contributions}

\begin{itemize}
    \item Estimate of duration of PCR-positivity: of interest for rest of thesis but also more broadly for public health (\eg interpreting PCR test results)
    \item Use in interpreting test results
    \item Incidence estimates from only CIS
    \item Transmission estimates from only CIS
\end{itemize}

\subsection{Methodological contributions}

\begin{itemize}
    \item Theory justifying the use of deterministic backcalculation in this setting. Already used in practice but without understanding when the approximation is good. In particular: all that is required is that the proportion of the population sampled is small, not that the prevalence or incidence is sufficiently large.
    \item Double interval censoring with arbitrary truncation and false negatives
    \item Combining ATACCC and CIS duration estimates
    \item SEIR model using only prevalence surveys
    \item Application within TRE
\end{itemize}

\subsection{Structure}

Outline the chapters and how they fit together

\end{document}