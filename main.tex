\documentclass[a4paper]{report}
\usepackage{graphicx}

\usepackage{amsmath}
\usepackage{amssymb}
\usepackage{bm}
\usepackage[T1]{fontenc}
\usepackage{hyperref}
\usepackage[utf8]{inputenc}
\usepackage{textcomp} % provide symbols
\usepackage{todonotes}
% \usepackage{microtype}

\def\reals{\mathbb{R}}
\def\nats{\mathbb{N}}
\def\sampSpace{\mathcal{X}}
\def\dist{\sim}
\DeclareMathOperator{\E}{\mathbb{E}}
\DeclareMathOperator{\V}{\mathbb{V}}
\DeclareMathOperator{\prob}{\mathbb{P}}
\DeclareMathOperator{\var}{\mathbb{V}}
\DeclareMathOperator{\indicator}{\mathbb{I}}
\DeclareMathOperator{\cov}{Cov}
\DeclareMathOperator{\Ber}{Bernoulli}
\DeclareMathOperator{\Bin}{Binomial}
\DeclareMathOperator{\Poi}{Poisson}
\def\matr{\bm}
\def\set{\mathcal}
\renewcommand{\vec}{\bm}

\newcommand{\citePersonalComms}{(personal communication)}


\title{Estimating the incidence of SARS-CoV-2 from prevalence data}
\author{Joshua Blake}
\date{\today}

\usepackage[backend=biber,style=authoryear,uniquename=false]{biblatex}

\usepackage{subfiles} % Best loaded last in the preamble
\addbibresource{references.bib}

\begin{document}

\maketitle

\chapter{Text to use}

\section{Data}
The CIS continually enrols individuals and tests them on an ongoing basis, following a
specified testing protocol. The protocol specifies that the initial five
tests are spaced every 7 days, then the spacing is every 28 days.
However, real-world considerations often lead to variations in test
times and missed tests. Our analysis focuses solely on the binary result
(positive or negative) of these tests, assuming no misclassification
bias.

The CIS (Coronavirus Infection Survey) is a longitudinal study on a representative sample of households.
A full description of the study can be found in the supplementary materials of [Pouwels et al. 2020](https://www.thelancet.com/journals/lanpub/article/PIIS2468-2667(20)30282-6/fulltext#supplementaryMaterial) or [the study protocol](https://www.ndm.ox.ac.uk/covid-19/covid-19-infection-survey/protocol-and-information-sheets).
In brief, households are invited to the study from databases held by the ONS.
The number invited was to satisfy various target number of individuals swabbing per fortnight; relevant to the discussion in this document is that the target increased from 2,500 by the end of May 2020 to 150,000 in October 2020.
Within a household, all indvidiuals aged 2 and over were invited to participate.
Once invited, an enrolement swab would be taken followed by 4 further weekly swabs (giving a total of 5 swabs on days 0, 7, 14, 21, 28 relative to enrolement) after which monthly swabs are taken.

During the period we consider, the total cohort size expands due to the continuous recruitment into the study.
Following this, the number then decreases and stabilises as those recruited to meet the October target transition from weekly onto monthly testing.
See figure \@ref(fig:test-schedules) for further detail.

\todo[inline]{Add some descriptive figures eg those in ~/COVID/ons-incidence/duration_estimation/reproducible-sims/reports/simulation-studies.pdf}

\section{Producing duration from ATACCC}

The ATACCC analysis produces a posterior distribution for some population-level parameters, $\phi$.
Taking a posterior sample for this, $\phi^{(i)}$, we can then draw the parameters which determine the viral
load trajectory for a random individual $j$, $\theta_j$, from which that individual's duration $d_j^{(i)}$ can be calculated ([see this Overleaf paper for further details](https://www.overleaf.com/9549594748zbvpgjgvmgsd)).
By drawing $\theta_1, \theta_2, \dots, \theta_n$ from the distribution $p(\theta_j \mid\phi^{(i)})$, the density of the duration distribution, $f_A(t)$, can be
calculated for the posterior sample $i$ as:
$$
f^{(i)}_A(t) = \frac{1}{n}\sum_{j=1}^n \mathbb{I} \left( t-0.5 < d_j^{(i)} < t+0.5 \right)
$$
Given $f_A(t)$, the cumulative density and hazard functions can be calculated.
Repeating this for $N$ posterior samples of $\phi$ gives the posterior distribution.
I have used $N = 1000$ and $n = 100,000$, 
Since implementing this solution, I have found that Monte Carlo integration to calculate durations given $\theta_i$ is more accurate by avoiding discretisation errors.
However, the results here do not use this for consistency.

\chapter{Survival analysis with perfect testing} \label{perf-test}

\subfile{cis-perfect-testing}

\chapter{Survival analysis with misclassification} \label{imperf-test}

\subfile{cis-imperfect-testing}


\end{document}