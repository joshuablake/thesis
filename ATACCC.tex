\documentclass[thesis.tex]{subfiles}

\begin{document}
\ifSubfilesClassLoaded{
  \setcounter{chapter}{3}
}

\chapter{Estimating duration using biomarkers} \label{ATACCC}


\Cref{E-inc-prev} showed that the duration of PCR positivity is a key parameter for estimating the incidence of SARS-CoV-2 infections.
The aim of this chapter is to estimate the duration of PCR positivity using the viral load data from the ATACCC study (described in\todo{ref section describing ATACCC}).
I model a biomarker, the log viral load, over the course of an infection.\todo{make sure log viral load and Ct values are introduced in chapter 2}
From this model, I derive the distribution of the duration of PCR positivity.

First, I will explain how biomarkers are typically modelled, both for SARS-CoV-2 and historically in the biostatistical literature (\cref{ATACCC:sec:background}).
Next, I describe the model of \textcite{hakkiOnset} in detail, which applies these methods to the ATACCC data (\cref{ATACCC:sec:hakki}) and before modifying its observation model to improve the inference (\cref{ATACCC:sec:observation-modification}).
This modification is based on an earlier analysis using a preliminary version of the dataset, described in \cref{ATACCC:sec:original-analysis}.
Finally, I convert the model and parameter estimates of the biomarker's trajectory to a duration of PCR positivity (\cref{ATACCC:sec:duration}).
I will refine this estimate in future chapters before its use in estimating transmission.

\section{Background} \label{ATACCC:sec:background}

Two major approaches have been taken to model SARS-CoV-2 viral load trajectories.
There are either using ordinary differential equations (ODEs) with a biological interpretation or using mathematical functions that are flexible enough to capture the dynamics.
The ODE-based approach uses parameters representing biological properties of the virus and host~\autocites[e.g.:][]{ejimaEstimation,keVivo,kimQuantitative,goncalvesTiming,perelsonMechanistic}.
Such models are rarely analytically-tractable and the system's evolution is computed using numerical methods.
% Both of these approaches have been used with other viruses and biomarkers.
The approach that is the focus of this chapter is using a piecewise linear model for the biomaker trajectory and random effects to allow for within-individual correlation.

Piecewise linear, random effect models have a long history of use for longitudinal data in biostatistics~\autocites{slateStatistical}.
Often, as here, they are used to model the temporal evolution of biomarkers.
A few example applications are: CD4 counts following HIV infection~\autocites{langeHierarchical,lynchPredicting}, parasite density in blood samples following malaria treatment~\autocites{fogartyBayesian}, and measurements of cognitive decline due to Alzheimer's disease~\autocites{bealisleBayesian}.

In these models, the function's parameters are clearly related to features of the biomarker's trajectory, such as the time from first detection to peak viral load, and the viral load at the peak.
The piecewise linear model is the most common in the context of SARS-CoV-2 viral load~\autocites{clearyUsing,kisslerViral,larremoreTest}
Here, the logarithm of viral load increases linearly, peaks instantaneously, and then decreases linearly; more flexible functions have also been suggested~\autocites{quiltyQuarantine}.
This model for the trajectory I adopt in this chapter, from \textcite{hakkiOnset}, is based on the piecewise log-linear model, however, they approximate the model so that the trajectory is continuous (see \cref{ATACCC:sec:hakki} for details).

Random effects represent the within-individual correlation within the population while borrowing information between individuals~\autocite[chapter 24]{lashModern}.
Measurements of viral load from the same individual are correlated.
A random effect means that these measurements share parameter(s), providing correlation.
However, the random effect(s) for each individual is drawn towards a common mean, allowing information on their value to be shared between individuals.
Following much of the literature referenced in this section, I will take a Bayesian approach in this chapter.
In the Bayesian context, random effect models are often referred to as hierarchical or multilevel models because they can be interpreted as placing a prior at two levels of a parameter hierarchy~\autocite[chapter 5]{gelmanBDA}.
% $\phi_j$ would be referred to as the bottom level of the hierarchy, with $\mu$ and $\sigma$ being the top level.
% More complex models could have more levels, for instance if the individuals are grouped into subpopulations.

% Random effects are used to capture the variation between individuals and the correlation between measurements within the same individual.
% In this context, a random effect means that parameter values are specific to each individual, rather than being a population average.
% The parameter's distribution across the population is assumed to be some known distribution, normally a normal or log-normal distribution~\autocite[chapter 24]{lashModern}.
% Specifically, if $\phi_j$ is the value of some parameter for individual $j$, $\mu_\phi$ is the population mean of $\phi$, and $\sigma_\phi$ is the population standard deviation, then we model $\phi_j$ as: $\phi_j \sim \N(\mu_\phi, \sigma_\phi^2)$ or $\phi_j \sim \LN(\mu_\phi, \sigma_\phi^2)$.
% $\mu_\phi$ and $\sigma_\phi$ are parameters; in a Bayesian setting, a prior is placed on them in the normal way.
% This generalises to a vector of parameters per individual, $\vec{\phi}_j$, in the obvious way.
% In this chapter, $\mu_\phi$ and $\sigma_\phi$ are the parameters of interest, because we want to generalise our model across the population.
% The $\phi_j$s, which tells us about the parameter only for individuals in our sample, are considered nuisance parameters.

\section{\texorpdfstring{\textcite{hakkiOnset}}{Hakki \etal (2022)}} \label{ATACCC:sec:hakki}

The model of \textcite{hakkiOnset} has two components: a latent viral load model and an observation model.
The viral load model produces a deterministic trajectory for the viral load of an individual, a function of the model parameters.
The observation model then describes how the PCR testing process produces noisy observations of this viral load.
The observation model includes the possibility of false positive and negatives, and the limit of detection of the PCR test.
\todo{Make sure limit of detection is introduced previously when I first talk about PCR}
% The limit of detection means that viral loads below a certain threshold are not detected by the PCR test.
% They can be represented as a censoring of the viral load at the limit of detection.

Several features of the \textcite{hakkiOnset} model can be motivated by the desire to apply Stan, a modern and convenient MCMC implementation of the NUTS algorithm (see\todo{ref wherever I end up describing NUTS}).
The first of these is using a continuous viral load trajectory rather than the more common piecewise linear model.
As NUTS uses gradient information, it is more efficient when the model is differentiable.
This leads to the following model for the viral load at time $t$, $v_i(t)$, in individual $i$ is shown in \cref{ATACCC:fig:viral-load-model} and given by:
\begin{align}
v_i(\tau) = v_{\max,i} \frac{a_i+b_i}{b_ie^{-a_i(\tau-\tau_{\max,i})} + a_ie^{b_i(\tau-\tau_{max,i})}}. \label{ATACCC:eq:viral-load}
\end{align}
Time, $\tau$, is continuous and defined uniquely for each individual such that $\tau = 0$ at the time that individual $i$ was first tested.
$v_{\max,i}$ is the maximum viral load, which occurs at time $\tau_{\max,i}$.
As $\lvert \tau - \tau_{\max} \rvert$ becomes large, the viral load trajectory tends towards exponential growth (before the peak) or decline (after the peak) in viral load at rates $a_i$ and $b_i$ respectively.
% Before the peak, this is because $b(\tau-\tau_{\max,i})$ is small and hence $a\exp(b(\tau-\tau_{\max,i}))$ is negligible.
% After the peak, this is because $b\exp(-a(\tau-\tau_{\max,i}))$ is negligible.
This happens quickly for realistic values of the parameters.
Therefore, the log viral load is approximately piecewise linear (see \cref{ATACCC:fig:viral-load-model} for a representative viral load trajectory).
$\vec{\phi}_i = (v_{\max,i}, a_i, b_i)^T$ are the random effects for individual $i$.
$\tau_{\max,i}$ is inferred independently for each individual because the time from first test to peak is not a biological concept shared between individuals.
\begin{figure}
  \centering \includegraphics{ATACCC/typical_trajectory}
  \caption[Example viral load model trajectory.]{Example viral load trajectory produced by \textcite{hakkiOnset}'s viral load model, using realistic parameter values. The increase in log viral load is approximately linear with a gradient of $a_i$, here 5. The decrease is also linear, with gradient $-b_i$, here -1.4. The peak is at time $\tau_{\max,i}$, here 0, and log viral load $\log v_{\max,i}$, here = 15. The smoothing (lack of linearity) near the peak is visible. \label{ATACCC:fig:viral-load-model}}
  \todo[inline]{Add a few lines to example plot showing the effect of different parameters, move parameters to key}
\end{figure}

PCR tests are noisy observations of the viral load.
A positive PCR test has a Ct value.
Ct values are linearly related to viral load (see\todo{ref intro section}).
A negative PCR test signals either the viral load is too low to be detected, \ie below the limit of detection, or the test is a false negative.
Positive PCR tests, especially those with low viral loads, could be false positives, for example caused by contamination.

The noisy observations are modelled by assuming each measurement of log viral load has an iid error distributed $\N(0, \sigma_v)$.
The measurements are left-censored at the limit of detection, $\vlod$, meaning that any observation which would give a log viral load of less than $\vlod$ is instead observed as negative.
False positives and negatives are modelled by allowing a test failure with probability $\rho$.
Test failures are modelled as producing a log viral load reading drawn from $\N(x_0, \sigma_0^2)$ where $x_0$ and $\sigma_0^2$ are parameters to be estimated.
These readings are censored in the same way as true readings.
A censored test failure represents a false negative, otherwise the test failure represents a false positive.
\textcite{hakkiOnset} uses a prior for $x_0$ centred on $\vlod$, which encodes a prior belief that false negatives and false positives occur with equal probability.

Let the observed log viral load for individual $i$ at time $t$ be $y_{i,t}$, with negative tests signified by $y_{i,t}=v_\text{lod}$.
Then, its likelihood is:
\begin{align}
p(y_{i,t} \mid \cdot) &= \begin{cases}
    \rho f_N(y_{i,t} \mid x_0, \sigma_0^2) + (1 - \rho) f_N(y_{i,t} \mid \log v_i(t), \sigma_v^2) & \text{if $y_{i,t} > v_\text{lod}$}\\
    \rho F_N(v_\text{lod} \mid x_0, \sigma_0^2) + (1 - \rho) F_N(\vlod \mid \log v_i(t), \sigma_v^2) & \text{if $y_{i,t} = v_\text{lod}$} \label{ATACCC:eq:hakki-likelihood}
\end{cases}
\end{align}
where $f_N(y \mid \mu, \sigma^2)$ and $F_N(y \mid \mu, \sigma^2)$ are the pdf and cdf at $y$ of a Normal distribution with mean $\mu$ and variance $\sigma^2$ respectively.
The first term on each line is the likelihood contribution from test failures, either a false positive (top line) or false negative (bottom line).
The second term on the first line is the likelihood of the observed log viral load from non-failed test results.
The second term on the bottom line is the likelihood of the negative (censored) tests from non-failed test results.

Conditional on all $v_i$s, the observations are assumed independent.
Because $v_i$ is a deterministic transform of $\vec{\phi}_i$ and $\tau_i$, the observations are independent conditional on the model parameters, and the full likelihood is simply the product of the likelihoods of each observation.

Individuals are split into groups based on their vaccination status.
The first group is unvaccinated and infected with a variety of pre-Alpha, Alpha, and Delta variants.
The second group is vaccinated and all were infected by the Delta variant.
My focus is on the first group since they best correspond to the infections in the time period used throughout this thesis, Autumn 2020.

Biological plausibility requires constraining $\vec{\phi}_i$ to positive real values.
Each $\phi_i$ is modelled as a log-normal random effects by using the hierarchical prior $\phi_i \dist \LN(\vec{\mu}_{\phi,g(i)}, \matr{\Sigma}_\phi)$ where $g(i)$ is the group of individual $i$.
$\vec{\mu}_{\phi,g} = (\mu_{v_{\max},g}, \mu_{a,g}, \mu_{b,g})^T$ is a vector of the population means of the log-parameters.
As the random effects have a log-normal, rather than normal, distribution $\exp(\vec{\mu}_{\phi,g})$ is the population median of the parameter value but not the mean.
The covariance matrix $\matr{\Sigma}_\phi$ controls the variability of the population and allows for the parameters to be correlated.
For example, an individual with a high peak viral load may also clear the virus slowly because both have an underlying cause of a weak immune system.
To improve sampling efficiency, the multivariate version of the non-centred parameterisation~\autocites{papaspiliopoulosGeneral,stanReparameterization} is used.
This can be expressed as follows.
\begin{align}
  \vec{\eta}_i &\dist \N(0, \matr{C}_\phi) \\
  \log \vec{\phi}_i &= \vec{\mu}_\phi + \vec{\sigma}_\phi \circ \vec{\eta}_i
\end{align}
where $\vec{\sigma}_\phi$ is a vector of the standard deviations of the parameters, $\matr{C}_\phi$ is a correlation matrix, and $\circ$ is element-wise multiplication.
These are defined such that, by construction, $(\matr{\Sigma}_\phi)_{ij} = (\vec{\sigma}_\phi)_i (\vec{\sigma}_\phi)_j (\matr{C}_\phi)_{ij}$.
Correlation matrices have diagonal elements all 1, therefore this reparameterisation has the same number of parameters as the original model.

\textcite{hakkiOnset} uses weakly informative priors, similar to those previously used~\autocite{singanayagamCommunity}.
These are given in \cref{ATACCC:table:hakki-priors}.
I reran \textcite{hakkiOnset}'s analysis using R~4.3.1~\autocite{R-4-3-1}.
The inference was performed using RStan~2.32.3~\autocite{RStan-2-32-3} and Stan~2.32.2~\autocite{Stan-2-32-2}.
Inference settings were the same as \textcite{hakkiOnset}, which included 8 chains with a total of 8000 iterations each with the first 3000 used as warm-up.
\begin{table}
\begin{tabular}{l c l l}
    Parameter description & Symbol & Distribution \\
    \hline \\
    Mean max viral load & $\mu_{v_\text{max}, g}$ & $\N(15, 15^2)$ \\
    Standard deviation of max log viral load & $\sigma_{v_\text{max}}$ & $\N_+(0, 10^2)$ \\
    Mean rate of viral load increase & $\mu_{a, g}$ & $\N(1.25, 0.75^2)$ \\
    Standard deviation of rate of viral load increase & $\sigma_{a}$ & $\N_+(0, 1)$ \\
    Mean rate of viral load decrease & $\mu_{b, g}$ & $\N(0.5, 1.4^2)$ \\
    Standard deviation of rate of viral load decrease & $\sigma_{b}$ & $\N_+(0, 1)$ \\
    Correlation of viral load parameters & $\matr{C}_\phi$ & $\LKJ(1)$ \\
    Negative log of rate of test failure & $-\log \rho$ & $\N(3, 0.5)$ \\
    Mean log viral load for test failures & $x_0$ & $\N(0, 3^2)$ \\
    Standard deviation of log viral load for test failures & $\sigma_0$ & $\N_+(3, 3^2)$ \\
    Standard deviation of log viral load observations & $\sigma_v$ & $1 + \N_+(2, 3^2)$ \\
    Time from first test to peak & $\tau_{\max,i}$ & $\N(5,5)$
\end{tabular}
\caption[Viral load model priors]{Priors for each parameter, taken from \textcite{hakkiOnset}. $\N_+$ indicates a normal distribution truncated to the non-negative reals. Details of the distributions, including their parameterisations, are in \cref{E-distributions}. The elements of $\vec{\mu}_{\phi,g}$ and $\vec{\sigma}_\phi$ have independent priors placed on them, as specified in the first lines of the table. Priors are the same for all $g$ and $i$. \label{ATACCC:table:hakki-priors}}
\end{table}

\section{Modifying the Hakki observation model} \label{ATACCC:sec:observation-modification}

The model of \textcite{hakkiOnset} did not converge fully (see \cref{ATACCC:sec:results}).
In particular, the Rhat values, traceplots, and effective sample size for the parameters of the observation model were poor.

Based on the fit to the data achieved when implementing an earlier version of the model (see \cref{ATACCC:sec:original-analysis}), I hypothesised that the cause of this was overcomplicating the model for test failures.
Specifically, the model excluding false positives explains the data well (the results in \cref{fig:paper:duration,ATACCC:fig:goodness-of-fits} do not allow false negatives but fit the data well).
Therefore, the model for test failures can be simplified to only false negatives.
% These false negatives could represent a variety of scenarios, such as poor swabbing technique.
Formally, the model can be expressed by replacing \cref{ATACCC:eq:hakki-likelihood} with the following.
\begin{align}
p(y_{i,t} \mid \cdot) &= \begin{cases}
    (1 - \rho) f_N(y_{i,t} \mid \log v_i(t), \sigma_v^2) & \text{if $y_{i,t} > v_\text{lod}$}\\
    \rho + (1 - \rho) F_N(\vlod \mid \log v_i(t), \sigma_v^2) & \text{if $y_{i,t} = v_\text{lod}$.}
\end{cases}
\end{align}

The parameters $x_0$ and $\sigma_0$ are no longer used in the model.
For all other parameters, I kept the priors the same as \textcite{hakkiOnset}, including the prior on $\rho$, and performed inference with the same settings (see end of \cref{ATACCC:sec:hakki}).


\section{Deriving duration from the parameters} \label{ATACCC:sec:duration}

The ultimate aim of this chapter is to produce an estimate of the distribution for the duration of PCR positivity.
So far, only the viral load trajectories have been considered; \textcite{hakkiOnset} does not consider this duration either.
The time that individual $i$ are first and last positive are the two values of $\tau$ such that $\log v_i(\tau) = v_\text{lod}$.
Denote these times as $\tau_{0,i}$ and $\tau_{1,i}$.
Hence, their duration of PCR positivity, in continuous time, is $d_i = \tau_{1,i} - \tau_{0,i}$.

At $\tau_{0,i}$ and $\tau_{1,i}$, the approximation as exponential growth (described in \cref{ATACCC:sec:hakki}) is good.
Therefore, set $\log v(\tau)$ and one of the exponential terms to 0 in \cref{ATACCC:eq:viral-load}.
When $v_\text{lod} = 0$, as it does here, this gives the following expressions:
$$
\begin{aligned}
\tau_{0,i}
&= \tau_{\max,i} - \frac{1}{a_i} \left(\log v_{\max,i} + \log(a_i + b_i) - \log b_i \right) \\
\tau_{1,i}
&= \tau_{\max,i} + \frac{1}{b_i} \left(\log v_{\max,i} + \log(a_i + b_i) - \log a_i \right) \\
d_i
&= \tau_{1,i} - \tau_{0,i}  \\
&= \frac{1}{a_i b_i} \left( (a_i + b_i) (\log v_{\max,i} + \log(a_i + b_i)) - a_i \log a_i - b_i \log b_i \right).
\end{aligned}
$$
Note that $d_i$ does not depend on $\tau_{\max,i}$, only $\phi_i$.

\todo[inline]{insert a better explanation of $F_D$'s posterior}
For later use, a discretised version is preferred (as described in \cref{E-inc-prev}).
To calculate the discrete version of this distribution, $F_D$, a Monte Carlo integration is used with the following relation.
\begin{align}
  F_D(d)
  &= \prob(D_i \leq d + 0.5) \\
  &= \E_{D_i}(\I(D_i \leq d + 0.5)) 
\end{align}
where $\I$ is the indicator function.
Here, $D_i$ is drawn conditional on the population-level parameters $\vec{\mu}_\phi$ and $\matr{\Sigma}_\phi$.
Equivalent forms of $F_D$, such as the survival and hazard functions, can be calculated using the relations in\todo{ref section defining survival functions}.
I found 100,000 draws produced stable estimates of the tail of $F_D$.

\section{Results} \label{ATACCC:sec:results-discussion}


The original \textcite{hakkiOnset} model has convergence issues (Rhat > 1.07 and ESS < 100) for the parameters of the test failure distribution, $\sigma_0$ and $\rho$.
The simplified model I proposed in \cref{ATACCC:sec:observation-modification} solved this issue, meaning that this analysis converged.

The modified model produced good fits to the data (see \cref{ATACCC:fig:goodness-of-fits}).
Almost all test results are captured in the 95\% credible intervals.
Individual 55 has some fitting issues, explored further below.
Individual 57 appears to have far noisier measurements than the other individuals, suggesting that the observation noise might vary between individuals.
Including a random effect on $\sigma_v$ could capture this affect.
However, it would complicate the model, be unlikely to change the estimates of the parameters of interest, and may not be identifiable.
\begin{figure}
  \thisfloatpagestyle{empty}
  \vspace{-3cm}
  \makebox[\textwidth][c]{\includegraphics[width=.9\paperwidth]{ATACCC/fits}}
  \captionsetup{width=0.8\paperwidth}
  \captionof{figure}[Posterior predictive viral load fits]{Posterior predictive observed viral loads (median and 95\% CrI, lines and ribbon) for each individual, overlaid with their observed test results (dots). Posterior estimates are conditional on the result not being a false negative. Negative PCR tests are plotted at the limit of detection, \ie a log viral load of 0. Each plot is labelled with the individual's position in the dataset, an arbitrary value. \label{ATACCC:fig:goodness-of-fits}}
\end{figure}

For individual-level parameters, both models had convergence issues in places (Rhat > 1.4 and ESS < 20).
However, this was greatly improved with the simplified model, occurring only for parameters relating to a single individual, number 55.
Individual 55 has an extremely noisy set of observations.
This means that two viral load trajectories fit the data well (shown in \cref{ATACCC:fig:individual-55}).
The first, found by two chains, increases very quickly, peaks 4.0 days after the first test, then declines slowly.
The second, found by the other six chains, is more symmetrical, peaking 7.2 days after the first test.
The population-level parameter estimates agree between the two sets of chains, therefore the issue should not affect any estimates of interest.
\begin{figure}
  \centering \includegraphics{ATACCC/fit_individual_55}
  \caption[Goodness-of-fit for individual 55]{Fit to individual 55's data coloured by MCMC chain. Two chains favour the quickly increasing viral load trajectory, peaking at day 4.0, the other six favour the slower trajectory, peaking at day 7.2. \label{ATACCC:fig:individual-55}}
\end{figure}

The two models give very similar estimates for the population-level parameters (see \cref{ATACCC:fig:compare-hakki-modified}).
These are the main parameters of interest (as explained in \cref{ATACCC:sec:duration}), suggesting that the results are robust to the choice of model.
As the estimates are similar and the modified model converges better, I use the modified model for the remainder of this chapter.
\begin{figure}
  \centering \includegraphics{ATACCC/compare_hakki_modified}
  \caption[Comparison of population-level parameters between models.]{Kernel-density smoothed posterior densities for the population mean (top), $\vec{\mu}_{\phi,1}$, and covariance matrix (bottom), $\matr{\Sigma}_\phi$, for both the \textcite{hakkiOnset} model (described in \cref{ATACCC:sec:hakki}) and the model with a modified observation model to improve convergence (described in \cref{ATACCC:sec:observation-modification}). Mean only shown for the unvaccinated group, the primary group of interest. The groups share the covariance matrix. These parameters, which are the primary ones of interest (see \cref{ATACCC:sec:duration}), have very similar estimates between the models. \label{ATACCC:fig:compare-hakki-modified}}
\end{figure}

Clear differences emerge between the vaccinated and unvaccinated groups (see \cref{ATACCC:fig:median-trajectories} and \cref{ATACCC:table:population-estimates}).
The rates of viral load increase at similar rates in the two groups, the vaccinated group have higher peak viral loads and clear the virus faster.
This leads to a shorter duration of PCR positivity in the vaccinated group.
\begin{figure}
  \centering \includegraphics{ATACCC/mean_trajectories}
  \caption[Median viral load trajectories]{Trajectory (posterior median and 95\% CrI) using the median parameter values, $\exp \vec{\mu}_{\phi,g}$, excluding the within-population variation. These assume $\tau_{\max,i}=0$, however modifying this simply translates the trajectories left or right. \label{ATACCC:fig:median-trajectories}}
\end{figure}
\begin{table}
  \makebox[\linewidth]{
  \begin{tabular}{lcc}
    Parameter & Unvaccinated population & Vaccinated population \\
    Max log viral load, $\mu_{v_\text{max},g}$ & 14.9 (14.2--15.7) & 16.0 (15.2--16.8) \\
    Rate of log viral load increase, $\exp(\mu_{a,g})$ & 5.07 (3.70--7.06) & 1.25 (1.18--1.32) \\
    Rate of log viral load decrease, $\exp(\mu_{b,g})$ & 1.33 (1.09--1.63) & 1.89 (1.50--2.36) \\
  \end{tabular}
  }
  \caption{Population-level estimates of the viral load parameters (posterior medians and 95\% CrI). The vaccinated population have higher peak viral loads but clear the virus faster. \label{ATACCC:table:population-estimates}}
\end{table}

The negative rate, $\rho$, is estimated to be quite low, 5.3\% (95\% CrI: 3.7--7.3\%).

The $F_D$ estimates for the two groups are in \cref{ATACCC:fig:duration}.
In \cref{ATACCC:sec:results-discussion}, the two models gave very similar results.
Therefore, I show results only from the modified model (which had better convergence).
\begin{figure}
  \centering \includegraphics{ATACCC/duration}
  \caption[Duration of PCR positivity.]{Estimated distribution (survival function) of duration of PCR positivity for the two populations. The vaccinated population have shorter durations of PCR positivity.  \label{ATACCC:fig:duration}}
\end{figure}

The vaccinated population have shorter durations of PCR positivity than the unvaccinated population (see \cref{ATACCC:fig:duration}).
The median duration of PCR positivity for the vaccinated population is 13 days (95\% CrI: 11--16 days), but it is 16 days (95\% CrI: 14--19 days) for the unvaccinated population.

\section{Discussion}

This chapter has estimated the distribution of PCR positive duration based on the viral load data from the ATACCC study.

Different estimates were produced for vaccinated and unvaccinated individuals in the study.
However, there are many differences between individuals in these two groups.
For example, the vaccinated individuals are likely to be older and/or have co-morbidities because the roll-out of the vaccine in the UK prioritised these groups~\autocite{naoCovidVaccination}.
They were also infected by different variants because the vaccinated individuals were infected later.
All these factors could change the viral dynamics~\autocite{russellWithinhost} and hence the duration.
Therefore, the relationships shown here cannot be considered causal, or even generalisable to contexts where these other confounders are not present.

A strength of the ATACCC data is that the sample is drawn from the general population.
This is unlike most estimates (see\todo{ref section reviewing lit on duration}) which are from hospitalised patients or healthcare workers.
This means that, notwithstanding the issues highlighted in the previous paragraph, the estimates are likely to be generalisable to the general population.

In the following chapters, I will use the estimates from the unvaccinated individuals.
The CIS data I use is from autumn 2020 and early 2021 when few individuals were vaccinated and the majority of infections were by pre-Alpha or Alpha variants.
Therefore, this group of people should be fairly representative of those within CIS.

A major issue with the estimates in this chapter is that they are extrapolated.
In ATACCC, individuals were tested for at most 20 days.
Yet, the estimates here sill have 33\% (95\% CrI: 21--46\%) of individuals positive at 20 days following infection and less than 1\% positive at 50 days.
The remainder of the estimate for $f_D$ is driven by the assumed form of the viral load trajectory.
However, the CIS has individuals testing positive for very long periods of time.
In periods of low incidence following a large peak, the number of individuals who are currently positive is dominated by those who have been positive for a long time.
Therefore, understanding if these probabilities are correct is important for understanding transmission in the period after a peak.
These effects are even more prominent when incidence falls suddenly to low levels due to an intervention such as a lockdown.
This is exactly the situation in the UK in early 2021.
The next two chapters will incorporate CIS data to improve the estimates of these tail probabilities.

\ifSubfilesClassLoaded{
  \listoftodos
  \appendix
  \subfile{distributions}
  \subfile{ATACCC-appendix-original-analysis}
  \listoftodos
}{}

\end{document}